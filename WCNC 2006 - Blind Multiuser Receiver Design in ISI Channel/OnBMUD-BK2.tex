\documentclass[a4paper,10pt,fleqn, twocolumn]{IEEETran}
\usepackage{amsfonts}
\usepackage{amsthm}
\usepackage{graphicx}
\usepackage{fancyhdr}
\usepackage{floatflt}

\newtheorem{Prop}{Proposition}
\newtheorem{lemma}{Lemma}

\setlength{\parindent}{3em} \setlength{\oddsidemargin}{0in}
\setlength{\textwidth}{6.5in} % sets 1in left and right margins
\setlength{\topmargin}{0.20in} % change to 0.2in for regular latex
%\setlength{\headheight}{0in}
%\setlength{\footheight}{0.5in}
\setlength{\footskip}{0.5in}
\setlength{\textheight}{9.0in} %sets 1in top and bottom margins
\renewcommand{\baselinestretch}{1} %set to 1.5 for double spacing.

\newcommand{\br}{{\mathbf r}}
\newcommand{\bA}{{\mathbf A}}
\newcommand{\ba}{{\bf a}}
\newcommand{\bb}{{\bf b}}
\newcommand{\bc}{{\bf c}}
\newcommand{\bC}{{\bf C}}
\newcommand{\bg}{{\bf g}}
\newcommand{\bG}{{\bf G}}
\newcommand{\bd}{{\bf d}}
\newcommand{\be}{{\bf e}}
\newcommand{\bq}{{\bf q}}
\newcommand{\bs}{{\bf s}}
\newcommand{\bm}{{\bf m}}
\newcommand{\bn}{{\bf n}}
\newcommand{\bu}{{\bf u}}
\newcommand{\bv}{{\bf v}}
\newcommand{\bw}{{\bf w}}
\newcommand{\bx}{{\bf x}}
\newcommand{\by}{{\bf y}}
\newcommand{\bz}{{\bf z}}
\newcommand{\bbf}{{\bf f}}
\newcommand{\bE}{{\bf E}}
\newcommand{\bF}{{\bf F}}
\newcommand{\bL}{{\bf L}}
\newcommand{\bM}{{\bf M}}
\newcommand{\bN}{{\bf N}}
\newcommand{\bS}{{\bf S}}
\newcommand{\bT}{{\bf T}}
\newcommand{\bD}{{\bf D}}
\newcommand{\bX}{{\bf X}}
\newcommand{\bP}{{\bf P}}
\newcommand{\bQ}{{\bf Q}}
\newcommand{\bI}{{\bf I}}
\newcommand{\bR}{{\bf R}}
\newcommand{\bU}{{\bf U}}
\newcommand{\bV}{{\bf V}}
\newcommand{\bW}{{\bf W}}
\newcommand{\bY}{{\bf Y}}
\newcommand{\bZ}{{\bf Z}}
\newcommand{\bJ}{{\bf J}}
\newcommand{\bB}{{\bf B}}
\newcommand{\bzero}{{\bf 0}}
\newcommand{\bgamma}{{\mbox {\boldmath $\gamma$}}}
\newcommand{\btheta}{{\mbox {\boldmath $\theta$}}}
\newcommand{\bLambda}{{\mbox {\boldmath $\Lambda$}}}
\newcommand{\bPsi}{{\mbox {\boldmath $\Psi$}}}
\newcommand{\bPhi}{{\mbox {\boldmath $\Phi$}}}
\newcommand{\bphi}{{\mbox {\boldmath $\phi$}}}
\newcommand{\bcA}{{\mbox {\boldmath ${\cal A}$}}}
\newcommand{\bcB}{{\mbox {\boldmath ${\cal B}$}}}
\newcommand{\bcC}{{\mbox {\boldmath ${\cal C}$}}}
\newcommand{\bcD}{{\mbox {\boldmath ${\cal D}$}}}
\newcommand{\bcF}{{\mbox {\boldmath ${\cal F}$}}}
\newcommand{\bcN}{{\mbox {\boldmath ${\cal N}$}}}
\newcommand{\bcR}{{\mbox {\boldmath ${\cal R}$}}}
\newcommand{\bcS}{{\mbox {\boldmath ${\cal S}$}}}
\newcommand{\bcH}{{\mbox {\boldmath ${\cal H}$}}}
\newcommand{\bcI}{{\mbox {\boldmath ${\cal I}$}}}


\title{On Blind Multiuser Detection}
\author{Shu Wang and James Caffery, Jr.}
\date{}
\begin{document}
\maketitle
\begin{abstract}\small
Multiuser signal model not only presents received signal structure
but also plays a key role in receiver design. In this paper, we
present a new blind multiuser signal model and discuss its
applications for blind multiuser receiver design. We compare it
with the popular conventional signal model and subspace signal
model and their blind multiuser receivers, respectively. After
this, the geometric interpellation, bit-error rate, signal
processing bounds, etc. of these signal models and blind receivers
are discussed and compared. Through these, the trade-offs between
the complexity and performance of blind multiuser receiver design
can be revealed. Computer simulations are finally provided to
demonstrate the performance and theoretic analysis.
\end{abstract}

\section{Introduction}
Multiuser detection (MUD) is the strategy for mitigating multiple
access interference (MAI) effects and solving the near-far problem
with exploiting interference structure. MUD has been extensively
investigated over the past several years, since MAI is the
dominant impairment for CDMA systems and even exists in the
systems with perfect power control~\cite{Verd98}. MUD is believed
to be one of the critical techniques for enabling the high
reliability and throughput of next-generation mobile communication
systems~\cite{Andr05}. Most recent research on MUD has been
devoted to blind detection and subspace-based signature waveform
or channel estimation for reducing the computation complexity and
prior knowledge~\cite{Honi95,Torl97,Wang98,Zhang02,Wang05B}. Blind
multiuser receivers can achieve good performance with the
knowledge of only desired user's timing and signature waveform.
This assumption also is much closer to practical applications
where most interfering signals are unknown beforehand at the
receiver. However, most existing blind receiver designs are known
to be too complicated to be implemented in high-data-rate
applications.

In the development of advanced multiuser receivers, it sees that
multiuser signal model not only provides us a description of
received signals but also help us understand the inner structure
of received signals, which is crucial for receiver design. There
are two popular multiuser signal models widely discussed for
multiuser receiver design. They are the conventional multiuser
signal model and subspace multiuser signal model. In the
conventional signal model, each received signal is directly taken
as a linear combination of actual signal
signatures~\cite{Verd98,Honi95,Zhang02}. And most related blind
multiuser receivers are developed either by explicitly estimating
the signal signature~\cite{Torl97} or by filtering interfering
signal components using some adaptive filter techniques, e.g., the
blind receiver design using Wiener filter~\cite{Honi95} and Kalman
filter~\cite{Zhang02} techniques. Though the conventional signal
model provides us a natural view of received signals, some
important information including signature waveforms or amplitudes
in this model is previously unknown and it will take the receiver
a lot of efforts to obtain it before detection. For compensating
this weakness of the conventional signal model, the subspace
signal model is proposed with subspace-based signal processing
techniques~\cite{Wang98}. In the subspace signal model, each
received signal is taken as a linear combination of signal
subspace bases, which is obtained by subspace signal processing
techniques on the autocorrelation matrix of received signals.
Subspace signal mode can be taken as a result of parametric signal
modelling, which provides a in-depth comprehension of received
signals. Though subspace-based approaches don't need explicitly
estimate each user's signal signature and the initialization and
adaptive speed can be much faster with good performance, the
signal subspace formation procedure still is not trivial.

It is known that the conventional signal model provides us the
foundation for both optimal and conventional multiuser receiver
design and subspace signal model helps us with understand received
signal structure for blind receiver design. However, either of
these two may not be easy enough for us to develop the blind
multiuser receivers for high-speed CDMA systems~\cite{Andr05}. In
order to solve the near-far problem with minimum prior knowledge
and computation complexity, we propose a new blind multiuser model
with directly connecting the current received signal and several
previous received signal and no explicitly signal structure
estimation. With this blind signal model and widely discussed
signal processing criteria, such as least squares (LS), minimum
mean-squared errors (MMSE) and maximum likelihood (ML), several
new blind multiuser receivers are developed to detect the desired
user's information. There is no statistical signal estimation or
subspace separation procedure. Only a minimum number of previously
received signals and the desired user's signal signature waveform
and timing are required. Hence the computation complexity and
detection delay can be much reduced. After this, we compare the
proposed blind signal model and receivers with the conventional
signal model and subspace signal mode and their blind receivers.
The trade-off between the performance and complexity in blind
receiver development is discussed too. Computer simulations are
finally presented.

\section{Multiuser Signal Models}
We consider forward-link transmissions in a single-cell DS/CDMA
system. There are $K$ active users over the multipath channel with
$P$ strong paths~\footnote{Strong paths are those to be explicitly
combined by RAKE receiver.} and the channel is an additive white
Gaussian noise (AWGN) channel. The baseband representation of the
received signal due to user $k$ is given by
\begin{equation}
\begin{array}{rcl}
r_k(t)&=&\sum\limits_{p=1}^{P}\alpha_{pk}A_k[n]
b_k[n]c_k(t-nT-\tau_p)
\end{array}
\end{equation}
\noindent where $\alpha_{pk}$ is the $p$th path loss of user $k$'s
signal, $b_k{[n]}$ is the $n$th bit sent by user $k$. We assume
that the $\left\{b_k{[n]}\right\}$ are independent and identically
distributed random variables with $E\left\{b_k{[i]}\right\}=0$ and
$E\left\{|b_k{[i]}|^2\right\}=1$. The parameters $c_k(t)$ denote
the normalized spreading signal waveform of user $k$ during the
interval $[0,\ T]$, $0\leq\tau_1\leq\tau_2\leq\ldots\leq\tau_P$,
denotes $P$ different transmission delays from the base station to
user $k$ and $A_k[n]$ is the received signal amplitude for user
$k$ at time $t=n$, which depends on the possible channel
statistics. The total baseband signal received by user $k$ is
\begin{equation}
\begin{array}{rcl}
\tilde{r}(t)&=&\sum\limits_{k=1}^{K}r_k(t)
\end{array}
\end{equation}
The received signal $\tilde{r}(t)$ is passed through the
corresponding chip matched filter (CMF), $\phi(t)$, and RAKE
combiner. The combined output $r(t)$ is~\footnote{Without loss of
the generality, we drop the time index $n$ in the following
discussion.}
\begin{equation}\hspace{-0.0in}
\begin{array}{rcl}
r(t)&=&A_k b_k c_k(t-nT-\tau_1)\otimes \phi(t-\tau_1)+ \\
&&\hspace{0.0in} m_{\rm ISI}(t) + m_{\rm MAI}(t) + n(t)
\end{array}\label{r_t}
\end{equation}
\noindent where
\begin{equation} \hspace{-0.05in}
\begin{array}{rcl}
 m_{\rm ISI}(t)&=&\\
 &&\hspace{-0.83in}\sum\limits^{P}_{p\neq
q}\beta_{qk} \alpha_{pk}A_kb_kc_k(t-nT+\tau_{q1}-\tau_1)\otimes
\phi(t-\tau_1)
\end{array}
\end{equation}
\noindent is the intersymbol interference (ISI) to user $k$,
\begin{equation} \hspace{-0.17in}
\begin{array}{rcl}
m_{\rm MAI}(t)&=&\sum\limits_{i\neq
 k}^{K}A_ib_ic_i(t-nT-\tau_1)\otimes\phi(t-\tau_1)+\\
 &&\hspace{-0.75in}\sum\limits_{i\neq
 k}^{K}\sum\limits^{P}_{p\neq
q}\beta_{qk}
\alpha_{pi}A_ib_ic_i(t-nT+\tau_{q1}-\tau_p)\otimes\phi(t-\tau_1)
\end{array}
\end{equation}
\noindent is the MAI to user $k$, $\beta_{qk}$ is the weight of
the $q$th RAKE finger with
$\sum\limits_{q=1}^{P}\beta_{qk}\alpha_{qk}=1$ and $\tau_{q1} =
\tau_{q}-\tau_1$ is the propagation delay difference between the
$1$st path and $p$th path. $\otimes$ denotes the convolutional
product. $n(t)$ is AWGN with variance $\sigma^2$. Because of
$m_{\rm MAI}(t)$ existing in the received signal $r(t)$, the
performance of conventional matched filter receiver suffers from
the so-called near-far problem~\cite{Verd98}. Multiuser detection
is one of the receiver techniques for solving this problem.

\subsection{\em Conventional Signal Model}
After RAKE combining, the user $1$'s output can be sampled at
$f_s=1/T_s$ and directly expressed by~\footnote{Without loss of
the generality, we name the first user as the desired user.}
\begin{equation}\hspace{-0.1in}
\begin{array}{rcl}
\br&=&\left[
\matrix{r(nT+T_s+\tau_1)&\ldots&r(nT+LT_s+\tau_1)}\right]^{\rm
T}\\
 &=&\sum\limits_{k=1}^{K} A_k b_k \bs_k + \bn \\
 &=&\bS \bA \bb + \bn
\end{array}\label{r_sync}
\end{equation}
\noindent where $\bS=[\bs_1\ \bs_2\ \ldots\ \bs_K]$ is the
received spreading signature matrix combined with inter-chip
interference (ICI), inter-symbol interference (ISI) and MAI
information, and $L=T/T_s$ is the number of samples per symbol,
which should not be less than the spreading gain $L_c$. Most MUD
schemes including optimum and conventional MUD are developed from
(\ref{r_sync}), which is named conventional multiuser signal
model. They are well documented in~\cite{Verd98}. One of the major
problems using (\ref{r_sync}) is $\left\{\bs_k,\
A_k:k\neq1\right\}$ or possible timing is unknown at receiver.
This may make multiuser receiver design complicated.

\subsection{\em Subspace Signal Model}
It is known that it is hard to accurately estimate the
$\left\{\bs_k:k\neq1\right\}$ in (\ref{r_sync}) in order to
directly apply the well-developed optimum or conventional
multiuser detection schemes. Another approach is to use
subspace-based signal model and signal processing techniques for
reconstruct the conventional detectors~\cite{Wang98}. In the
subspace signal model, $\br$ is modelled by the combination of the
signal subspace bases $\left\{\bu_{sk}:1\leq k\leq K\right\}$:
\begin{equation}
\begin{array}{rcl}
\br&=&\bU_{s}\bphi+\bn
\end{array}\label{r_ss}
\end{equation}
\noindent where $\bU_{s}=\left[\bu_{s1}\ \bu_{s2}\ \ldots\
\bu_{sK}\right]$, $\bphi$ is a vector defined by
\begin{equation}
\begin{array}{rcl}
\bphi&=&\bPhi\bA\bb
\end{array}
\end{equation}
\noindent with $\bPhi$ is a $K\times K$ matrix. The original
signal signature matrix $\bS$ can now be expressed by
\begin{equation}
\begin{array}{rcl}
\bS&=&\bU_{s}\bPhi
\end{array}.
\end{equation}
\noindent One of the major advantages of the subspace signal model
(\ref{r_ss}) is that the signal subspace bases
$\left\{\bu_{sk}:1\leq k\leq K\right\}$ are much easier to be
estimated than the actual signal signature waveforms so that the
blind receiver design can be much simplified. These signal bases
can be estimated applying subspace decomposition on the
autocorrelation matrix $\bR$:
\begin{equation}
\begin{array}{rcl}
\bR &=&{\rm E}\{\br\br^{\rm T}\}\\
&=&\left[\matrix{\bU_{s}&\bU_{n}}\right]\left[\matrix{\bLambda_{s}&\cr&\bLambda_{n}}\right]\left[\matrix{\bU_{s}^{\rm
T}\cr\bU_{n}^{\rm T}}\right]
\end{array}.\label{R}
\end{equation}
\noindent where $\bU_{n}$ denotes the noise subspace bases and
$\left[\cdot\right]^{\rm T}$ denotes the transportation operation.

\subsection{\em The Proposed Blind Signal Model}
As we can see, one of the difficulties in using the conventional
signal model or subspace signal model for blind receiver design is
the signal signatures $\left\{\bs_k:k\neq1\right\}$ in
(\ref{r_sync}) or the signal subspace matrix $\bU_{s}$ in
(\ref{r_ss}) are unknown beforehand. Instead we propose a known
blind signature matrix $\bcS$
\begin{equation}
\begin{array}{rcl}
\bcS&=&[\matrix{\bs_1&{\br}_{1}&{\br}_{2}&\ldots&{\br}_{M-1}}]\\
&=&\bS\bD + {\bN}
\end{array} \label{S_0}
\end{equation}
\noindent where $\left\{\br_{m}:1\leq m\leq M-1\right\}$ are
previously received and detected signal vectors and
\begin{equation}\hspace{-0.0in}
\begin{array}{c}
 \bD=\left[\matrix{1 & \bar\bd^{\rm T}\cr\bzero&\tilde\bD }\right]=\left[\matrix{\be & \matrix{\bar\bd^{\rm T}\cr \tilde{\bD}} }\right]
  =\left[\matrix{\bg^{\rm T} \cr \matrix{\mathbf{0}& \tilde{\bD}}
 }\right]
\end{array}\label{D}
\end{equation}
\noindent is the $K\times M$ data matrix associated with $\bcS$.
Now the received signal can be expressed by
\begin{equation}
\begin{array}{rcl}
\br&=&\bcS\bbf + \tilde{\bn} \label{r_blind}
\end{array}
\end{equation}
\noindent with the $M \times 1$ vector $\bbf$ defined by
\begin{equation}
\begin{array}{rcl}
\bbf&=&\bD^{+}\bA\bb
\end{array}, \label{DetectorVector}
\end{equation}
\noindent where $\tilde{\bn}$ is the new $L\times 1$ AWGN vector
defined by
\begin{equation}
\begin{array}{rcl}
\tilde{\bn}&=&\bn-{\bN}\bD^{+}\bA\bb
\end{array} \label{new_noise}
\end{equation}
\noindent $\be=[\matrix{1&\bzero}]^{\rm T}$ is a vector of length
$K$, $\bg = \left[\matrix{1&\bar\bd}\right]^{\rm T}$ is a vector
of length $M$.

Different to the models in (\ref{r_sync}) and (\ref{r_ss}), the
key component $\bcS$ in (\ref{r_blind}) is known beforehand. This
makes this model ready for designing new blind multiuser receivers
from the beginning. After we estimate $\bbf$ with conventional
signal estimation techniques, the detection of user $1$'s
information can easily be done with knowing $d_1=A_1b_1$, which
can be estimated by
\begin{equation}
\begin{array}{rcl}
\hat{d}_{1} &=&\bg^{\rm T}\bbf
\end{array}.\label{d_1}
\end{equation}

\section{Blind Multiuser Receiver}
\subsection{\em Conventional Blind Multiuser Detection}
With the conventional signal model in (\ref{r_sync}), there are
two popular directions for designing blind multiuser receivers.
One is to estimate the unknown $K-1$ signal signatures
$\left\{\bs_k:k\neq1\right\}$ and then apply known
optimal/conventional detectors on $\br$. This approach is know to
be computation-intensive since the signal waveform estimation
itself is not simple~\cite{Torl97}. The other one is to use
adaptive filter techniques with some signal processing criteria.
Under this direction, there are well-known minimum output energy
(MOE) detector, blind MMSE detector and blind Kalman
detector~\cite{Honi95,Verd98,Zhang02}. However, these blind
detectors are known to be slow in their adaptive procedures.

\subsection{\em Subspace-Based Blind Multiuser Detection}
With the subspace signal mode in (\ref{r_ss}), the first step
usually is to separate signal/noise subspaces and estimate
$\bU_{s}$ using (\ref{R}). After this, the least-square-based
decorrelating detector is given by
\begin{equation}
\begin{array}{l}
\bw_{\rm DD1}=\frac{1}{\bs_1^{\rm
T}\bU_{s}\left(\bLambda_{s}-\sigma\bI\right)^{-1}\bU_{s}^{\rm
T}\bs_{1}}\bU_{s}\left(\bLambda_{s}-\sigma\bI\right)^{-1}\bU_{s}^{\rm
T}\bs_{1}
\end{array}
\end{equation}
\noindent and the MMSE detector can be expressed by
\begin{equation}
\begin{array}{rcl}
\bw_{\rm MMSE1}&=&\frac{1}{\bs_1^{\rm
T}\bU_{s}\bLambda_{s}^{-1}\bU_{s}^{\rm
T}\bs_{1}}\bU_{s}\bLambda_{s}^{-1}\bU_{s}^{\rm T}\bs_{1}
\end{array}
\end{equation}
\subsection{\em New Blind Multiuser Detection Approaches}
With the blind signal model in (\ref{r_blind}), one possible
approach is firstly to estimate the detection vector $\bbf$.
Obviously different signal estimation criteria may lead to
different solutions. If the LS criterion is direct applied here,
the traditional LS estimation of $\bbf$ can be expressed by
\begin{equation}
\begin{array}{rcccl}
{\bbf}_{\rm
LS}&=&\matrix{\mbox{arg}\min\limits_{\bx}\left\|\br-\bcS\bx\right\|_2}&=&\bcS^+\br
\end{array}
\label{LSProb}
\end{equation}
\noindent with the assumption that $\bcS$ is error-free and
$d_{1}$ can be estimated by
\begin{equation}
\begin{array}{rcl}
{d}_{\rm LS1}&=&\bg^{\rm T}\bcS^+\br
\end{array}. \label{d_LS}
\end{equation}
\noindent If $\bcS$ is assumed to be error-polluted in the above
LS estimation problem (\ref{LSProb}), we have the total least
squares (TLS) problem
\begin{equation}
\begin{array}{l}
\left[\matrix{\bcS_{\rm TLS}\cr\bbf_{\rm
TLS}}\right]=\matrix{\mbox{arg}\min\limits_{\bar{\bcS},\
\bx}\left\|\left[ \matrix{\bcS\cr\br} \right] - \left[
\matrix{\bar{\bcS}\cr\bar{\bcS}\bx}\right]\right\|_2}
\end{array}.
\label{TLSProb}
\end{equation}
\noindent With solving (\ref{TLSProb}), the TLS estimation of
$\bbf$ can be written by
\begin{equation}
\bbf_{\rm TLS} = \left(\bcS^{\rm
T}\bcS-\sigma_{K+1}^2\bI\right)^{-1}\bcS^{\rm T}\br
\end{equation}
\noindent and $d_{1}$ can be estimated by
\begin{equation}
\begin{array}{rcl}
{d}_{\rm TLS1}&=&\bg^{\rm T}(\bcS^{\rm
T}\bcS-\sigma_{K+1}^2\bI)^{-1}\bcS^{\rm T}\br
\end{array}. \label{b_TLS}
\end{equation}
\noindent If only the first column of the $\bcS$ in (\ref{LSProb})
is assumed to be error-free, we have the mixed least-square (MLS)
problem:
\begin{equation}
\begin{array}{l}
\left[\matrix{\bcS_{\rm MLS}\cr\bbf_{\rm
MLS}}\right]=\matrix{\mbox{arg}\min\limits_{\bar{\bcS},\
\bx}\left\|\left[\matrix{\tilde{\bcS}\cr\br}\right]-\left[\matrix{\bar{\bcS}\cr[\bs_1\
 \bar{\bcS}]\bx}\right]\right\|_{2} }
\end{array},\label{MLSProb}
\end{equation}
\noindent With solving (\ref{MLSProb}), the MLS estimation of
$\bbf$ can be written by
\begin{equation}\hspace{-0.09in}
\begin{array}{rcl}
\bbf_{\rm MLS}&=&\left(\bcS^{\rm
T}\bcS-\sigma^2\left[\matrix{\bzero&\mathbf{0}\cr\mathbf{0}&\mathbf{I}_{M-G}}\right]\right)^{-1}\bcS^{\rm
T}\br
\end{array}
\end{equation}
\noindent and $d_{1}$ can be estimated by
\begin{equation}\hspace{-0.13in}
\begin{array}{l}
{d}_{\rm MLS1}=\bg^{\rm T}\left(\bcS^{\rm
T}\bcS-\sigma^2\left[\matrix{0&\mathbf{0}\cr\mathbf{0}&\mathbf{I}_{M-1}}\right]\right)^{-1}\bcS^{\rm
T}\br
\end{array}. \label{b_MLS}
\end{equation}
If the minimum variance unbiased (MVU) criterion is applied,
$\bbf$ can be estimated by
\begin{equation}
\begin{array}{rcl}
{\bbf}_{\rm MVU}&=&\left(\bcS^{\rm
T}\bR_{\tilde{\bn}}^{-1}\bcS\right)^{-1}\bcS^{\rm
T}\bR_{\tilde{\bn}}^{-1}\br
\end{array} \label{BLUE}
\end{equation}
\noindent and $d_{1}$ can be estimated by
\begin{equation}\hspace{-0.0in}
\begin{array}{l}
{d}_{\rm MVU1}=\bg^{\rm T}\left(\bcS^{\rm
T}\bR_{\tilde{\bn}}^{-1}\bcS\right)^{-1}\bcS^{\rm
T}\bR_{\tilde{\bn}}^{-1}\br
\end{array}, \label{b_MLS}
\end{equation}
\noindent where
$\bR_{\tilde{\bn}}=\mbox{E}\left\{\tilde{\bn}\tilde{\bn}^{\rm
T}\right\}$ denotes the autocorrelation matrix of $\tilde{\bn}$.

If the MMSE criteria is applied, $\bbf$ can be estimated by
\begin{equation}\hspace{-0.07in}
\begin{array}{rcl}
{\bbf}_{\rm MMSE}&=&\mbox{arg}\min\limits_{\hat{\bbf}}\mbox{E}\|\hat{\bbf}-\bbf\|_2^2\\
&=&\left(\bR_{\bbf}^{-1}+\bcS^{\rm
T}\bR_{\tilde{\bn}}^{-1}\bcS\right)^{-1}\bcS^{\rm
T}\bR_{\tilde{\bn}}^{-1}\br
\end{array} \label{MSE}
\end{equation}
\noindent and $d_{1}$ can be estimated by
\begin{equation}\hspace{-0.05in}
\begin{array}{l}
d_{\rm MMSE1}=\bg^{\rm T}\left(\bC_{\bbf}^{-1}+\bcS^{\rm
T}\bC_{\bar{\bn}}^{-1}\bcS\right)^{-1}\bcS^{\rm
T}\bC_{\bar{\bn}}^{-1}\br
\end{array}\label{b_MMSE}
\end{equation}

\section{Performance Analysis and Comparison}

\subsection{Geometrical Interpellation} It is known that
conventional decorrelating detector can be interpolated as an
oblique projection of $\bs_1$ onto the orthogonal complement of
the signal subspace $\bar\mathbb{S}_{1}=\mbox{span}\left\{\bs_k :
k=2,\ 3,\ \ldots ,\ K\right\}$ along the orthogonal complement of
$\mathbb{S}_{1}=\mbox{span}\left\{\bs_1\right\}$~\cite{Elda02}
while conventional blind MMSE detection can be taken as a balance
between single-user matched filter and decorrelating detector.
Obviously the subspace based decorrelating detector and MMSE
detector have the same geometrical interpellation with the
conventional detectors, respectively.

For the proposed blind least-square receiver, the projection using
$\bcS$ can be interpolated as an oblique projection of $\bs_1$
onto the orthogonal complement of the signal subspace
$\tilde\mathbb{S}_{1}=\mbox{span}\left\{\br_m : k=1,\ 2,\ \ldots
,\ M-1\right\}$, instead of $\bar\mathbb{S}_{1}$, along the
orthogonal complement of $\mathbb{S}_{1}$. Since
$\tilde\mathbb{S}_{1}\neq\bar\mathbb{S}_{1}$, there is some
deviation between this project and the previous projection using
$\bar\mathbb{S}_{1}$ and therefore there is some error between
$d_1$ and the first element of $\bbf$. Fortunately, this error can
be compensated with (\ref{d_1}).

\subsection{AME and Near-Far Resistance} A commonly used
performance measure for a multiuser detector is asymptotic
multiuser efficiency (AME) and near-far resistance~\cite{Verd98}.
Since the proposed algorithms converge to the conventional
decorrelating detector as $\sigma^2\rightarrow 0$, their AME and
near-far resistance are identical to the decorrelating detector:
\begin{equation}
\begin{array}{rcl}
\bar{\eta}_k&=&\left[\bR_{\bs}^{+}\right]_{kk}^{-1}
\end{array}
\end{equation}
\noindent where $\bR_{\bs}=\bS^{\rm T}\bS$ is called signature
autocorrelation matrix and $\left[\cdot\right]_{kk}$ denotes the
$k$th diagonal element of the matrix.

\subsection{Noise Enhancement} It is known that there is a noise
enhancement issue in LS-based decorrelating detection. With
conventional decorrelating detection, the output signal-to-noise
ratio (SNR) for user $k$ is decreased by
$\left[\bR_{\bs}^{+}\right]_{kk}^{-1}$. For the proposed blind LS
multiuser receiver, there is another noise enhancement issue.
Following Girko's law, providing $\alpha=\frac{K-1}{M}$ is fixed,
the diagonal element of
$\frac{1}{M}\left(\bD^{+}\bb\right)\left(\bD^{+}\bb\right)^{\rm
T}$ can be approximated to be $1-\alpha$ with $K$, $M$
$\rightarrow\infty$~\cite{Muller}. And the covariance matrix of
$\tilde\bn$ can be expressed by
\begin{equation}
\begin{array}{rcl}
\bR_{\tilde\bn}&=&\frac{2M+K-1}{M}\sigma^{2}\bI
\end{array}.\label{noise_var_new}
\end{equation}
\noindent Therefore the noise variance is increased by
$\frac{2M+K-1}{M}$.
\subsection{Cram\'{e}r-Rao Lower Bound} The Cram\'{e}r-Rao lower
bound (CRLB) is given by the inverse of the Fisher information
matrix (FIM). For the conventional signal model, CRLB of
$\bd=\bA\bb$ estimation is given by
\begin{equation}
\begin{array}{rcl}
{\rm CRLB}(\bd\ |\ \bS)&=&\sigma^{2}\left(\bS^{\rm
T}\bS\right)^{\rm -1}
\end{array}.\label{CRLB_b}
\end{equation}
\noindent It shows that the conventional decorrelating detection
can achieve maximum likelihood sense if the amplitude $\bA$ is
unknown. For the blind signal model, if the blind spreading matrix
$\bcS$ is known beforehand, we first define the parameter vector
$\mathbf{\psi} = \left[\tilde{\sigma}^{2}\ \bbf^{\rm
T}\right]^{\rm T}$, where $\tilde{\sigma}^{2}
=(1+\frac{M-1}{M-K})\sigma^{2}$, for computing the FIM
\begin{equation}
\begin{array}{rcl}
{\bI(\mathbf{\phi})} &=& {\rm E} \left\{ \left( \frac{\partial
\ln{\cal L}}{\partial \mathbf{\psi}} \right) \left( \frac{\partial
\ln{\cal L}}{\partial \mathbf{\psi}} \right)^{\rm T} \right\}
\label{fim}
\end{array}
\end{equation}
\noindent where $\ln{\cal L}$ is the log-likelihood function given
by
\begin{equation}
\begin{array}{rcl}
\ln{\cal
L}&=&C-L\ln\tilde{\sigma}^2-\frac{1}{2\tilde{\sigma}^2}\parallel\mathbf{e}\parallel_2^2
\end{array},\label{logl}
\end{equation}
\noindent $C$ is a constant and $\mathbf{e}=\br-\bcS\bbf$.
Providing $\bcS$ is known, the closed-form CRLB expression of
$\bbf$ is then given by
\begin{equation}
\begin{array}{rcl}
{\rm CRLB}(\bbf\ |\ \bcS) &
=&(1+\frac{M-1}{M-K})\sigma^{2}(\bcS^{\rm T}\bcS)^{\rm +}
\end{array}.\label{CRLB_f}
\end{equation}
\noindent It shows that the accuracy of estimating $\bbf$ may
increase with increasing $M$.

\subsection{Bit-Error Rate}
With decorrelating detection, the bit-error rate (BER) for user
$k$ is
\begin{equation}
\begin{array}{rcl}
P_{ek,\rm\small DD} &=&
Q\left(\frac{E_{k}}{\sigma^{2}\left[\bR_{\bs}^{+}\right]_{kk}}\right)
\end{array},\label{BER_DD}
\end{equation}
\noindent where $E_{k}$ is the bit energy of user $k$. Similarly
for the blind LS multiuser receiver, the output BER can be written
by
\begin{equation}\hspace{-0.07in}
\begin{array}{rcl}
P_{e1,\rm\small LS} &=&
Q\left(\frac{E_{k}}{\tilde\sigma^{2}\left[\bR_{\tilde\bs}^{+}\right]_{11}}\right)\\
&=&
Q\left(\frac{ME_{k}}{\left(2M+K-1\right)\sigma^{2}\left[\left(\bcS^{\rm
T}\bcS\right)^{+}\right]_{11}}\right)\ ,
\end{array}\label{BER_LS}
\end{equation}
\noindent where $\bR_{\tilde\bs}=\bcS^{\rm T}\bcS$.

\section{Computer Simulations}
\section{Conclusions}
In this paper, we present another view on blind multiuser receiver
design with proposing a new blind signal model and several related
new blind multiuser receivers. We compared the proposed signal
model and receivers with the existing conventional signal model
and subspace signal model and related receivers side by side in
terms of geometric properties, asymptotic multiuser efficiency and
near-far resistance, noise enhancement, Cram\'{e}r-Rao lower
bound, etc., and discussed the trade-off between performance and
complexity.

\small
\bibliographystyle{unsrt}
\bibliography{FastBDD,InterferenceCancellation}
\end{document}

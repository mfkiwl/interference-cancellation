
\documentclass[a4paper,11pt,fleqn]{article}
\usepackage{amsfonts}
\usepackage{amsthm}
\usepackage{graphicx}

\setlength{\parindent}{3em} \setlength{\oddsidemargin}{0in}
\setlength{\textwidth}{6.5in} % sets 1in left and right margins
\setlength{\topmargin}{0.0in} % change to 0.2in for regular latex
%\setlength{\headheight}{0in}
%\setlength{\footheight}{0.5in}
\setlength{\footskip}{0.5in}
\setlength{\textheight}{9.0in} %sets 1in top and bottom margins
%\renewcommand{\baselinestretch}{1} %set to 1.5 for double spacing.


\newtheorem{Prop}{Proposition}
\newtheorem{lemma}{Lemma}

\newcommand{\br}{{\mathbf r}}
\newcommand{\bA}{{\mathbf A}}
\newcommand{\ba}{{\bf a}}
\newcommand{\bb}{{\bf b}}
\newcommand{\bc}{{\bf c}}
\newcommand{\bC}{{\bf C}}
\newcommand{\bg}{{\bf g}}
\newcommand{\bG}{{\bf G}}
\newcommand{\bd}{{\bf d}}
\newcommand{\be}{{\bf e}}
\newcommand{\bs}{{\bf s}}
\newcommand{\bm}{{\bf m}}
\newcommand{\bn}{{\bf n}}
\newcommand{\bu}{{\bf u}}
\newcommand{\bv}{{\bf v}}
\newcommand{\bw}{{\bf w}}
\newcommand{\bx}{{\bf x}}
\newcommand{\bbf}{{\bf f}}
\newcommand{\bF}{{\bf F}}
\newcommand{\bL}{{\bf L}}
\newcommand{\bM}{{\bf M}}
\newcommand{\bN}{{\bf N}}
\newcommand{\bS}{{\bf S}}
\newcommand{\bT}{{\bf T}}
\newcommand{\bD}{{\bf D}}
\newcommand{\bX}{{\bf X}}
\newcommand{\bP}{{\bf P}}
\newcommand{\bQ}{{\bf Q}}
\newcommand{\bI}{{\bf I}}
\newcommand{\bR}{{\bf R}}
\newcommand{\bU}{{\bf U}}
\newcommand{\bV}{{\bf V}}
\newcommand{\bW}{{\bf W}}
\newcommand{\bJ}{{\bf J}}
\newcommand{\bB}{{\bf B}}
\newcommand{\bzero}{{\bf 0}}
\newcommand{\bgamma}{{\mbox {\boldmath $\gamma$}}}
\newcommand{\btheta}{{\mbox {\boldmath $\theta$}}}
\newcommand{\bLambda}{{\mbox {\boldmath $\Lambda$}}}
\newcommand{\bPsi}{{\mbox {\boldmath $\Psi$}}}
\newcommand{\bPhi}{{\mbox {\boldmath $\Phi$}}}
\newcommand{\bcA}{{\mbox {\boldmath ${\cal A}$}}}
\newcommand{\bcS}{{\mbox {\boldmath ${\cal S}$}}}
\newcommand{\bcH}{{\mbox {\boldmath ${\cal H}$}}}
\newcommand{\bcI}{{\mbox {\boldmath ${\cal I}$}}}
\newcommand{\bcR}{{\mbox {\boldmath ${\cal R}$}}}
\newcommand{\bcB}{{\mbox {\boldmath ${\cal B}$}}}

\title{Blind Multiuser Detection Algorithms For CDMA Downlinks}
\date{}
\author{Research \& Standards Group \\ LG Mobilephone USA, Inc.}
\begin{document}
\maketitle

\begin{abstract}
Multiuser detection is one of the key techniques for combating
multiple access interference (MAI) in CDMA systems. In this paper,
we propose a new blind multiuser detection framework with a new
system model and then several new blind detection algorithms based
on this framework using least-squares-based (LS) estimations, best
least unbiased (BLU) estimation, minimum mean-squared error (MMSE)
estimation and Kalman filtering estimation criteria. Theoretical
analysis and computer simulations are provided to demonstrate the
proposed schemes too.
\end{abstract}

\section{Introduction}

Direct-sequence code division multiple access (DS/CDMA) techniques
have attracted increasing attention for efficient use of available
bandwidth, resistance to interference and flexibility to variable
traffic patterns. Multiuser detection strategy is a method to
minimize the effect of MAI and solve the near-far problem in CDMA
systems. It has been extensively investigated over the past
several years~\cite{Verd98}, since MAI is the dominant impairment
for CDMA systems and exists even in perfect power-controlled CDMA
systems. Most early work on multiuser detection assumed that the
receiver knew the spreading codes or had some knowledge of all
users, then exploited this knowledge to combat MAI. For example,
the classic decorrelating detector can achieve the optimum
near-far resistance and completely eliminate MAI from other users
with the expense of enhancing background noise. However, in many
practical cases, especially in a dynamic environment, e.g. in the
downlink of the CDMA system, it is difficult for a mobile user to
obtain accurate information on other active users in the same
channel. On the other hand, the frequent use of training sequence
is certainly a waste of channel resource. So blind multiuser
detection has been proposed. Recent research has been devoted to
blind multiuser receivers and subspace-based signature waveform
estimation schemes to achieve better performance and higher
capacity~\cite{Madh94,Honi95, Poor97, Wang98, Torl97, Liu96}. The
Minimum output energy (MOE) method and subspace method were
presented for multiuser blind detection with the knowledge of only
the desired users' spreading code and possible timing.

Blind multiuser detectors can achieve good performance with the
knowledge of the time and signature waveform of desired user while
they don't require intensive computation, compared with many other
conventional multiuser detectors. The blind adaptive multiuser
detectors in~\cite{Madh94,Honi95} are based on the minimization of
MMSE between the outputs and data. The asymptotic form of MMSE
multiuser detector is the same as the conventional decorrelating
detector. It is shown that the MOE receiverin~\cite{Honi95} is
equivalent to the linear MMSE detector, which is near-far
resistant and has much less complexity compared the optimal
multiuser detection. The major limitation of MOE schemes to
multiuser blind detection is that there is a satuaration effect in
the steady state, which causes a significant performance gap
between the converged blind MOE and the true MMSE
detector~\cite{Honi95}.

Blind multiuser detection using subspace techniques was first
developed by Wang and Poor~\cite{Wang98, Poor98}. Such techniques
were appropriate for the downlink environment where only the
desired users' code is available. More recently, these subspace
techniques were extended by Wang and Host-Madsen~\cite{Wang99},
named group multiuser blind detectors, to uplink environments
where the base station knows the codes of in-cell users, but not
those of users outside the cell. In the subspace-based blind
detection approach~\cite{Wang98}, the linear detectors are
constructed in the closed form once the signal subspace components
are computed and offer lower computational complexity and better
performance than the blind MOE detector. For the subspace-based
blind adaptive detector, the project approximation subspace
tracking deflation (PASTd) algorithm~\cite{Yang95} is used to
estimate the signal subspace.


As we see, various multiuser detection schemes have been developed
to mitigate the effects of MAI. Blind detectors are obviously much
closer to practical applications. So far, most blind detectors are
based on the classic multiuser system model in~\cite{Verd98}. They
normally either employ some converging procedure based on some
criteria or restore other users's spreading sequence or the signal
or noise subspace before detecting desired information bits. This
is because only the desired user own spreading information is
known to the receiver in the widely-used classic system model. In
this work, we proposed an alternative new blind system model for
each individual user. Different to the classic system model, the
proposed blind spreading sequence matrix, which is constructed
using this user's known spreading sequence and several previously
received signal vectors, is already known to each user. Since
there is no other user's spreading sequence or amplitude
information in it, this is a blind model. Based on this blind
system model, we propose a blind multiuser detection framework and
several blind multiuser detectors with this framework. In our
work, we show that the bits sent for the desired user can be
clearly detected if there is no noise or distortion in the
proposed blind spreading sequence matrix in one step. When the
proposed spreading matrix is corrupted in practical situations,
many estimation schemes are proposed to blindly detect desired
information bits. These schemes include LS-liked estimations, BLU
estimation, linear MMSE estimation and Kalman filtering
estimation. In the present multiuser blind detectors, only the
signature and timing of the desired user are utilized. There is no
adaptive or search procedures employed as in other multiuser blind
detection algorithms. Theoretical analysis and computer
simulations are also presented to demonstrate the performance of
these blind detectors.

The rest of the paper is organized as follows. In Section II, we
summarize the signal model. In Section III, we propose a new blind
multiuser model and present the blind multiuser detection
framework. In Section IV, various estimation schemes are discussed
for blind detection. Performance analysis and simulation results
are provided in Section V and VI. Section VII concludes this
papers.


\section{Classic Multiuser System Model And Problem Description}

The basic CDMA $K$-user channel model, consisting of the sum of
antipodally modulated synchronous signature waveforms embedded in
additive white Gaussian noise (AWGN), is considered here. The
received base-band signal during one symbol interval in such a
channel can be modelled as:

\begin{equation}
\matrix{r(t)&=&\sum\limits_{k=1}^{K}A_k b_k [n]s_k (t)+ n(t)}
\end{equation}

\noindent where $t\in [nT,\ (n+1)T]$, $T$ is the symbol interval.
$n(t)$ represents the Gaussian channel noise, $K$ is the number of
users and $A_k$, $b_k[n]$ denote the received amplitude and data
bit of the $k$th user, respectively. It is assumed that
$b_k[n]\in\{-1,\ +1\}$ is a collection of independent equiprobable
$\pm1$ random variables transmitted by the $k$th user during
$[nT,\ (n-1)T]$ and $s_k(t)$ denotes the normalized signal
waveform of the $k$th user on the interval $[(n-1)T,\ nT]$, i.e.,
$\|s_k(t)\|=1$. The received signal $r(t)$ is passed through a
chip-matched filter followed by a chip-rate sampler. As a result,
$r(t)$, $t\in [(n-1)T,\ nT]$, is converted into a $L\times 1$
column vector $\br$~\footnote{Without the loss of generality, we
drop parameter $n$ and denote $\br=\br[n]$. The same to $b_k$} of
the samples of the chip-matched filter outputs within a symbol
interval $[(n-1)T,\ nT]$ as

\begin{equation}
\begin{array}{rcl}
\br &=& \left[\matrix{r_{{1}} & r_{{2}} & \ldots & r_{{L}}}\right]^{\rm T}\\
 &=& \sum\limits_{k=1}^{K} A_k b_k \bs_k + \bn \\
 &=& \bS \bA \bb + \bn
\end{array} \label{r}
\end{equation}

\noindent where $\bA=\mbox{diag}\left\{\left[\matrix{A_1\ A_2\
\ldots\ A_K}\right]\right\}$ is the received amplitude diagonal
matrix, $\bS = \left[\bs_1\ \bs_2\ \ldots\ \bs_K\right]$ is the $L
\times K$ signature matrix with the $k$th column $\bs_k$ being the
signature vector of the $k$th user, $\bb = [b_1\ b_2\ \ldots\
b_K]^{\rm T} = [b_1\ \tilde{\bb}^{\rm T}]^{\rm T}$ is the
information vector sent by all the $K$ users at time $t=n$ and
$b_1$ is the bit sent by the first user at time $t=n$, and $\bn$
is an $L$-dimensional Gaussian vector with independent
$\sigma^2$-variance components, $[\cdot]^{\rm T}$ is the
transposition operator. We maintain the restriction that $L \geq
K$.

Most of the linear multiuser detectors for demodulating the $k$th
user's data bit in (\ref{r}) is in the form of a correlator
followed by a hard limiter, which could be expressed as

\begin{equation}
\begin{array}{rcl}
\hat{b}_k &=& \mbox{sign}\{\bw_k^{\rm T}\br\}
\end{array} \label{linear}
\end{equation}

\noindent where $\bw_k \in \mathbb{R}^{L\times 1}$ is the linear
representation of multiuser detector. Linear multiuser detectors
can be implemented in a decentralized fashion where only the user
or users of interest need be demodulated.



\section{Blind Multiuser System Model And Detection Framework}

Most classic multiuser detection schemes assume knowledge of the
spreading waveform, amplitude and/or signal-to-noise ratios (SNR)
of all active users that contribute to received signals. These
receivers then exploit this knowledge to achieve optimal or
sub-optimal performance. These classic multiuser detectors have
comprehensively been investigated in~\cite{Verd98}. However, in
many practical situations, e.g. in the downlinks of CDMA mobile
systems, it is difficult for multiuser receiver to known other
existing users' information. So many blind multiuser detectors are
developed to operate without prior knowledge regarding other users
but using advanced signal processing and estimation techniques to
estimate other's information or reconstruct signal or noise
subspace. In this paper, instead of doing this, we take a
different approach for blind multiuser detection. At first, we
propose a new blind multiuser system model using a "faked" blind
spreading matrix for the desired user. We say it is a blind
"faked" spreading sequence matrix because 1) it only consists of
the desired user's spreading sequence and several previous
received signal vector and 2) it isn't a true original spreading
sequence matrix and there is no other active users' original
spreading sequences or amplitude in it. We then show that the
desired user's next information bits can be detected based on this
blind multiuser system model. In the following, we present this
novel blind multiuser system model and detection framework.
Without loss of the generality, assume only the bits $b_1$ sent by
the $1$st user is considered here.

\subsection{Blind Multiuser System Model}
At first we construct a new $L\times M$ blind spreading sequence
matrix $\bcS$ which is defined by

\begin{equation}
\begin{array}{rcl}
\bcS&=&[\matrix{\bar{\bs}_1&\bar{\bs}_2&\bar{\bs}_3&\ldots&\bar{\bs}_M}]\\
 &=&[\matrix{\bs_1&{\br}_{1}&{\br}_{2}&\ldots&{\br}_{M-1}}]\\
 &=&\left[\matrix{\bS\bcA\be_1&\bS\bcA{\bar\bb}_1&\bS\bcA{\bar\bb}_2&\ldots&\bS\bcA{\bar\bb}_{M-1}}\right] + {\bN}\\
 &=&\bS\bcA\left[\matrix{\be_1&{\bar\bb}_1&{\bar\bb}_2&\ldots&{\bar\bb}_{M-1}}\right] + {\bN}\\
 &=&\bS\bcA\left[\matrix{\be_1 & \bD }\right] + {\bN}\\
 &=&\bS\bcA\bB + {\bN}
\end{array} \label{S}
\end{equation}

\noindent where ${\br}_i$, $i=1,\ 2,\ \ldots,\ M-1$, are the
arbitrary previously received independent vectors and the $K\times
1$ vector ${\bb}_i=\left[\matrix{A_1b_1[i] & b_2[i]& \ldots &
b_K[i]}\right]^{\rm T} $ is the corresponding vector for the
information sent by all $K$ users, $M\geq K$.
$\bcA=\mbox{diag}\left\{\left[\matrix{1\ A_2\ \ldots\
A_K}\right]\right\}$, $\bD = [A_1{\bd}\ \tilde{\bD}^{\rm T}]^{\rm
T}$ and the $(K-1)\times 1 $ vector ${\bd}$ is the known bit
vector previously sent for the $1$st user. ${\bN}=[\mathbf{0}\
\tilde{\bN}]$ is a new noise matrix where the first column is a
zero vector and other columns are AWGN vectors with the same
distribution,

\begin{equation}
\begin{array}{rcl}
 \bB&=&\left[\matrix{\be_1 & \bD }\right]\\
  &=&\left[\matrix{\bg^{\rm T} \cr \matrix{\mathbf{0}& \tilde{\bD}}
 }\right]\\
 &=&\left[\matrix{\be_1 & \matrix{ A_1{\bd}^{\rm T}\cr\tilde{\bD}} }\right]\\
 &=&\left[\matrix{1& A_1\bd^{\rm T} \cr \mathbf{0}& \tilde{\bD}
 }\right]

\end{array}\label{B}
\end{equation}

\noindent are information matrix associated with the blind
spreading sequence matrix $\bcS$. $\bg = \left[\matrix{1&
A_1\bd^{\rm T}}\right]^{\rm T}$ is a known information vector,
$\mbox{rank}\{\tilde{\bD}\}=K-1$ and $\mbox{rank}\{\bB\}=K$.

Using the received signal vector definition (\ref{r}) and the
proposed blind signature matrix $\bcS$ in (\ref{S}), the received
signal vector $\br$ can be expressed as

\begin{equation}
\begin{array}{rcl}
\br&=&\bS\bcA\bar\bb + \bn\\
 &=&\bS\bcA\bB\bB^{+}\bar\bb + \bn\\
 &=&(\bcS-{\bN})\bB^{+}\bar\bb+ \bn\\
 &=&\bcS\bB^{+}
 \bar\bb - {\bN}\bB^{+}\bar\bb + \bn\\
 &=&\bcS\bbf + \bar{\bn} \label{rn}
\end{array}
\end{equation}

\noindent where $\bar\bb = [A_1b_1\ b_2\ \ldots\ b_K]^{\rm T} =
[A_1b_1\ \tilde{\bb}^{\rm T}]^{\rm T}$, $\bB^{+}  $ is the
Moore-Penrose matrix inverse of $\bB$ and $\bbf$ is termed the $K
\times 1$ detection vector defined by

\begin{equation}
\begin{array}{rcl}
\bbf&=&\left[\matrix{f_1\cr\tilde{\bbf}}\right]\\
 &=&\bB^{+}\bar\bb\\
 &=&\left[\matrix{\be_1&A_1\bD}\right]^{+}\bar\bb\\
 &=&\left[\matrix{1&A_1{\bd}^T\cr\mathbf{0}&\tilde{\bD}}\right]^{+}\left[\matrix{A_1b_1\cr\tilde{\bb}}\right]
\end{array} \label{DetectorVector}
\end{equation}

\noindent with the mean vector $\bm_{\bbf}=\bzero$ and the
covariance matrix $\bC_{\bbf}=\mbox{E}\left\{\bbf \bbf^{\rm
T}\right\}$, $\bar{\bn}$ is the new $L\times 1$ noise vector
defined by

\begin{equation}
\begin{array}{rcl}
\bar{\bn}&=&\bn-{\bN}\bB^{+}\bar\bb
\end{array} \label{new_noise}
\end{equation}

\noindent with mean vector $\bm_{\bar{\bn}}=\bzero$ and covariance
matrix

\begin{equation}
\begin{array}{rcl}
\bC_{\bar{\bn}}&=&\mbox{E}\left\{(\bn-{\bN}\bB^{+}\bar\bb)(\bn-{\bN}\bB^{+}\bar\bb)^{\rm T}\right\}\\
&=& \sigma^{2}\bI+\mbox{E}\left\{{\bN}\bbf\bbf^{\rm T}{\bN}^{\rm T}\right\}\\
&=&\sigma^2\left(1-\mbox{E}\|\tilde{\bbf}\|_2^2\right)\bI\ ,
\end{array} \label{var_n}
\end{equation}

\noindent and $\mbox{E}\left\{\cdot\right\}$ is the expectation
operator.


\subsection{Blind Multiuser Detection Framework} Before giving
blind multiuser detection framework, we show a semiblind multiuser
detection framework providing that the amplitude $A_1$ of the
$1$st user is already known. At this time, $b_1$ can be estimated
using the following theorem.


\begin{lemma}
The bit $b_1$ sent for the first user at time $t=n$ can be
detected by
\begin{equation}
\begin{array}{rcl}
b_1 &=& \mbox{sign}\left\{\bg^{\rm T}\bbf\right\}
\end{array}.
\end{equation} \label{bn_estimation}
\end{lemma}


\begin{proof}

At first, $b_1$ can be estimated by
\begin{equation}
\begin{array}{rcl}
b_1&=&\mbox{sign}\left\{b_1-{\bd}^{\rm T}\tilde{\bD}^{+}\tilde{\bb}+{\bd}^{\rm T}\tilde{\bD}^{+}\tilde{\bb}\right\}\\
&=& \mbox{sign}\left\{\left[\matrix{A_1^{-1}&{\bd}^{\rm
T}}\right]\left[\matrix{A_1b_1-A_1{\bd}^{\rm
T}\tilde{\bD}^{+}\tilde{\bb}\cr\tilde{\bD}^{+}\tilde{\bb}}\right]\right\}\\
&=& \mbox{sign}\left\{\bg^{\rm T}\left[\matrix{A_1&-A_1{\bd}^{\rm
T}\tilde{\bD}^{+}\cr\bzero&\tilde{\bD}^{+}}\right]\left[\matrix{b_1\cr\tilde{\bb}}\right]\right\}\\
&=& \mbox{sign}\left\{\bg^{\rm T}\left[\matrix{A_1&-A_1{\bd}^{\rm
T}\tilde{\bD}^{+}\cr\bzero&\tilde{\bD}^{+}}\right]\bb\right\}
\end{array}\label{b1}
\end{equation}


\noindent On the other hand, with (\ref{B}), we know that

\begin{equation}
\bB\left[A_1\be_1\ \matrix{-A_1{\bd}^{\rm
T}\tilde{\bD}^+\cr\tilde{\bD}^+}\right]\bB
=\left[\matrix{A_1^{-1}&{\bd}^{\rm
T}\cr\bzero&\tilde{\bD}}\right]\left[\matrix{A_1&-A_1{\bd}^{\rm
T}\tilde{\bD}^+\cr\bzero&\tilde{\bD}^+}\right]\left[\matrix{A_1^{-1}&{\bd}^{\rm
T}\cr\bzero&\tilde{\bD}}\right] =\bB\ . \label{psedoB1}
\end{equation}

\noindent So, the Moore-Penrose Matrix inverse $\bB^{+}$ of $\bB$
can be written by

\begin{equation}
\bB^{+} =\left[\matrix{A_1&-A_1{\bd}^{\rm
T}\tilde{\bD}^+\cr\bzero&\tilde{\bD}^+}\right]\ . \label{psedoB2}
\end{equation}


\noindent Using (\ref{B}) and (\ref{psedoB2}), (\ref{b1}) can be
re-written by

\begin{equation}
\begin{array}{rcl}
b_1&=& \mbox{sign}\left\{\bg^{\rm T}\bB^{+}\bb\right\}\\
&=& \mbox{sign}\left\{\bg^{\rm T}\bbf\right\}\\
\end{array}
\end{equation}

\noindent and the detection vector $\bbf$ can be re-written by

\begin{equation}
\begin{array}{rcl}
\bbf&=&\left[\matrix{A_1b_1-A_1{\bd}^{\rm
T}\tilde{\bD}^{+}\tilde{\bb}\cr\tilde{\bD}^{+}\tilde{\bb}}\right]
\end{array} \label{DetectorVector2}
\end{equation}


This lemma is finally proven.
\end{proof}


Now, the classic multiuser detection model (\ref{r}) is
transferred into (\ref{rn}) with the information bit vector $\bb$
being replaced by the detection vector $\bbf$ in
(\ref{DetectorVector2}) and the original AWGN vector $\bn$ being
replaced by $\bar{\bn}$ in (\ref{new_noise}). Fortunately, the bit
$b_1$ sent for user 1 can still be detected using the detection
vector $\bbf$, the previously detected bits vector $\bg$ and Lemma
\ref{bn_estimation}. However, since the first element of $\bbf$
and $\bg$ both depend on the amplitude $A_1$ of the $1$st user, we
call this framework the semiblind multiuser detection framework.
There are many schemes for estimating $A_1$. In the following, we
show that the amplitude $A_1$ can be estimated too if we know how
to estimate $\bbf$. How to estimate the detection vector $\bbf$
will be intensively discussed in Section \ref{LBD}.


Before estimating $A_1$, we construct another blind spreading
matrix $\bcS'$ using another set of received vectors, $\br_m'$. If
we have already known how to estimate the detection vector, we can
get two detection vectors, $\bbf$ and $\bbf'$, corresponding to
these two blind spreading matrices $\bcS$ and $\bcS'$. With $\bbf$
and $\bbf'$, $A_1$ can be estimated using the following theorem.

\begin{lemma}
$A_1$ is the solution to the following equation

\begin{equation}
\begin{array}{rcl}
\bg^{\rm T}\bbf - \bg'^{\rm T}\bbf'&=&0
\end{array}.\label{gf}
\end{equation}


\noindent and, if ${\bd'}^{\rm T}\tilde{\bbf}'\neq {\bd}^{\rm
T}\tilde{\bbf}$, it can be estimated by

\begin{equation}
\begin{array}{rcl}
{A}_1&=&{ f_{1} - f'_{1} \over {\bd'}^{\rm T}\tilde{\bbf}' -
{\bd}^{\rm T}\tilde{\bbf}}
\end{array}
\end{equation}

\noindent where $\bg'=\left[\matrix{A_1^{-1}& \bd^{\rm '
T}}\right]^{\rm T}$ and $\bbf'=[f'_{1}\ \tilde{\bbf}']^{\rm T}$.
\end{lemma}

\begin{proof}

(\ref{gf}) can be re-written by

\begin{equation}
\begin{array}{rcl}
(A_1^{-1}f_{1}+\bd^{\rm T}\tilde{\bbf})- (A_1^{-1}f'_{1}+\bd'^{\rm
T}\tilde{\bbf}')&=&0
\end{array}.
\end{equation}

\noindent When ${\bd'}^{\rm T}\tilde{\bbf}'\neq {\bd}^{\rm
T}\tilde{\bbf}$, $A_1$ can be estimated by $A_1={ f_{1} - f'_{1}
\over {\bd'}^{\rm T}\tilde{\bbf}' - {\bd}^{\rm T}\tilde{\bbf}}$.


 This lemma is then proven.
\end{proof}

\section{Linear Blind Multiuser Detectors\label{LBD}}

We can see that the performance of this blind multiuser detection
frame work highly depends on the estimation of $\bbf$ and $A_1$
which can be estimated from $\bbf$ too. With (\ref{rn}), the
estimation of $\bbf$ can be a typical linear estimation problem.
In the following, we discuss various algorithms for estimating
$\bbf$ using different criteria, which including LS, BLU, MMSE
criteria. The formulations of different blind multiuser detectors
are presented following each estimation scheme.


\subsection{Least-Squares-Based Blind Detections}
Let us start from minimizing least-squared errors. LS-based
algorithms are widely used in practices because they are simple
and easy to be implemented. They only use signal models without
any probabilistic assumptions about data. The negative side, no
claims about optimality can be made and the statistical
performance cannot be assessed without some specific assumption
about the probabilistic structure of the data.

\subsubsection{ Least Squares Detection }
At first, we assume the measurements of $\bcS$ is assumed to be
free of error. All errors are confined to the received vector
$\br$. Hence, the detection vector can be estimated with solving
the following equation

\begin{equation}
\begin{array}{rcl}
{\bbf}_{\rm
LS}=\matrix{\mbox{arg}\min\limits_{\bx}\left\|\br-\bcS\bx\right\|_2}&\mbox{subject
to}&\br\subseteq \mathbb{R}(\bcS)
\end{array}
\label{LSProb}
\end{equation}

Suppose $\bU^T\bcS\bV=\mathbf{\Sigma}$ is the SVD of
$\bcS\in\mathbb{R}^{L\times
 K}$ with $r=\mbox{rank}(\bcS)$. And if $\bU=[\matrix{\bu_1&\bu_2&\ldots&\bu_L}]$,
 $\bV=[\matrix{\bv_1&\bv_2&\ldots&\bv_K}]$, $\mathbf{\Sigma}=\mbox{diag}\{[\matrix{\sigma_1&\ldots\sigma_r&0&\ldots&0}]\}$ and $\br\in \mathbb{R}^{L\times 1}$,
 then LS estimation of $\bbf$ is

 \begin{equation}
 \begin{array}{rcccl}
 \matrix{\bbf_{\rm
 LS}&=&\sum\limits_{i=1}^{r}\frac{\bu_i^T\br}{\sigma_i}\bv_i&=&\bcS^+\br}&=&\bbf + \bcS^+\bar{\bn}
 \end{array}
 \end{equation}

\noindent which minimizes $\|\bcS\bd-\br\|_2$ and has the smallest
2-norm of all minimizers. Moreover
 \begin{equation}
 \matrix{\varepsilon_{\rm LS}^2 &=& \min\limits_{\bx\in\mathbb{R}}\|\bcS\bx-\br\|_2^2 &=& \sum\limits_{i=r+1}^{L}(\bu_i^T\br)^2}
 \end{equation}

\noindent With the estimated amplitude $\hat{A}_1$, the linear
filter representation of the least squares detector can be written
by

\begin{equation}
\begin{array}{rcl}
\bw_{\rm LS}&=&\bcS^{+\rm T}\bg
\end{array}. \label{w_LS}
\end{equation}

\subsubsection{Total Least Squares Estimation }

It assume $\bcS$ to be error-free in the previous LS estimate of
the detection vector $\bbf$. However, this assumption is not
entirely accurate according to the definition of $\bcS$ in
(\ref{S}) since there is a noise term, $\bN$. On the other hand,
$\br$ can also be expressed as
\begin{equation}
\begin{array}{rcl}
\br&=&(\bcS-\bN)\bB^{+}\bb + \bn\\
 &=&\hat{\bcS}\bd + \bn
\end{array}
\end{equation}
where  $\hat{\bcS}=\bcS-\bN=\bS\bA\bB$.  The minimization problem
of (\ref{LSProb}) can then be transformed into the following TLS
problem:
\begin{equation}
\begin{array}{rcl}
\left[\bcS_{\rm TLS},\ \bbf_{\rm
TLS}\right]&=&\matrix{\mbox{arg}\min\limits_{\bar{\bcS},\
\bx}\left\|\left[ \matrix{\bcS&\br} \right] - \left[
\matrix{\bar{\bcS}& \bar{\bcS}\bx}\right]\right\|_2}
\end{array},
\label{TLSProb}
\end{equation}
subject to $\br\subseteq\mathbb{R}(\bar{\bcS})$.

 Let $\bcS=\bU^{'}\mathbf{\Sigma}^{'}\bV^{'T}$ and
$[\bcS\ \br]=\bU\mathbf{\Sigma}\bV^T$ be the SVD of $\bcS$ and
$[\bcS\ \br]$, respectively. If $\sigma_K^{'}
> \sigma_{K+1}$, TLS estimation of $\bbf$ then is
\begin{equation}
\bbf_{\rm TLS} =
\left(\bcS^T\bcS-\sigma_{K+1}^2\bI\right)^{-1}\bcS^T\br
\end{equation}
and
\begin{equation}
\begin{array}{rcl}
\varepsilon_{\rm TLS}^{2}&=&\min\limits_{\bx\in
\mathbb{R}^{K\times
1}}\|\bcS\bx-\br\|_2^2 \\
 &=& \sigma_{K+1}^2\left[1+\sum_{i+1}^{K}\frac{\left(\bu_i^{'T}\br\right)^2}
{\sigma_i^{'2}-\sigma_{K+1}^2}\right]
\end{array}
\end{equation}
where $\bU=[\bu_1\ \bu_2\ \ldots\ \bu_L]$, $\bV=[\bv_1\ \bv_2\
\ldots\ \bv_{K+1}]$, $\mathbf{\Sigma}={\rm diag}\{[\sigma_1\
\sigma_2 \ldots\ \sigma_{K+\min\limits\{L-K,\ 1\}}]\}$ and
$\bU^{'}=[\bu_1^{'}\ \bu_2^{'}\ \ldots\ \bu_L^{'}]$,
 $\bV^{'}=[\bv_1^{'}\ \bv_2^{'}\ \ldots\ \bv_{K}^{'}]$,
 $\mathbf{\Sigma}^{'}={\rm diag}\{[\sigma_1^{'}\ \sigma_2^{'}\ \ldots\
 \sigma_K^{'}]\}$. And the linear filter $\bw_{\rm TLS}$ is

\begin{equation}
\begin{array}{rcl}
\bw_{\rm TLS}&=&\bcS(\bcS^{\rm T}\bcS-\sigma_{K+1}^2\bI)^{-1}\bg
\end{array}
\end{equation}

\subsubsection{Mixed LS/TLS Estimation}

In the LS problem of (\ref{LSProb}), it assumed the blind
signature matrix $\bcS$ is error-free. Again, this assumption is
not completely accurate. In the TLS problem of (\ref{TLSProb}), it
assumed that in each column of the blind signature matrix, $\bcS$,
some noise or error exists.  This assumption also is not complete.
Though there exists a noise or error matrix $\bN$ in $\bcS$ from
(\ref{S}), its first column is exactly known to be noise-free or
error-free.  Hence, to maximize the estimation accuracy of the
detection vector $\bbf$, it is natural to require that the
corresponding columns of $\bcS$ be unperturbed since they are
known exactly. The problem  of estimating the detection vector
$\bd$ can then be transformed into the following MLS problem by
considering (\ref{LSProb}) and (\ref{TLSProb}):
\begin{equation}
\begin{array}{rcl}
\left[\bcS_{\rm MLS},\ \bbf_{\rm MLS}\right] &=
&\matrix{\mbox{arg}\min\limits_{\bar{\bcS},\
\bx}\left\|\left[\matrix{\tilde{\bcS}&\br}\right]-\left[\matrix{\bar{\bcS}&[A_1\bs_1\
 \bar{\bcS}]\bx}\right]\right\|_{2} }
\end{array}\label{MLSProb}
\end{equation}
subject to $\br\subseteq\mathbb{R}([A_1\bs_1\ \bar{\bcS}])$.  The
following lemma outlines the MLS solution.

Consider the MLS problem in (\ref{MLSProb}) and perform the
Householder transformation $\bQ$ on the matrix
$[\matrix{\bcS&\br}]$ so that
\begin{equation}
\begin{array}{rcl}
\bQ^{\rm
T}[\matrix{A_1\bs_1&\bar{\bcS}&\br}]&=&\left[\matrix{R_{11}&\bR_{12}&R_{1r}\cr
\mathbf{0}&\bR_{22}&\bR_{2r}}\right]
\end{array}
\end{equation}
where $R_{11}= A_1$, $\bR_{12}$ is a $1\times (M-1)$ vector,
$\bR_{22}$ is a $(L-1)\times (M-1)$ matrix and $\bR_{2r}$ is a
$(L-1)\times 1$ vector and

\begin{equation}
\begin{array}{rcl}
\bQ &=&\bI - \frac{1}{A_1(A_1-s_{11})}\left[\matrix{s_{11}-A_1\cr
s_{12}\cr \vdots \cr s_{1L}}\right]\left[\matrix{(s_{11}-A_1)&
s_{12}& \cdots & s_{1L}}\right]
\end{array}
\end{equation}



Denote $\sigma'$ as the smallest singular value of $\bR_{22}$ and
$\sigma$ as the smallest singular value of
$[\matrix{\bR_{22}&\bR_{2r}}]$. If $\sigma'>\sigma$, then the MLS
solution uniquely exists and is given by
\begin{equation}
\begin{array}{rcl}
\bbf_{\rm
MLS}&=&\left(\bcS^T\bcS-\sigma^2\left[\matrix{0&\mathbf{0}\cr\mathbf{0}&\mathbf{I}_{M-1}}\right]\right)^{-1}\bcS^T\br
\end{array}.
\end{equation}

\noindent The linear filter representation of this MLS detection
is

\begin{equation}
\begin{array}{rcl}
\bw_{\rm
MLS}&=&\bcS\left(\bcS^T\bcS-\sigma^2\left[\matrix{0&\mathbf{0}\cr\mathbf{0}&\mathbf{I}_{M-1}}\right]\right)^{-1}\bg
\end{array}.
\end{equation}




\subsection{ Best Linear Unbiased Estimation}

We begin by assuming the following linear structure ${\bbf}_{\rm
BLU}=\bW_{\bbf}\br$ for this so-called best linear unbiased
estimator (BLUE). Matrix $\bW_{\bbf}$ is designed such that: 1)
$\bcS$ must be deterministic, and 2) $\bar{\bn}$ must be zero mean
with positive definite known covariance matrix $\bC_{\bar{\bn}}$,
3) ${\bbf}_{BLU}$ is an unbiased estimator of $\bbf$, and 4) the
error variance for each of the $M$ parameters is minimized as

\begin{equation}
\begin{array}{rcl}
\bW_{\bbf}&=&\min\limits_{\bW} \mbox{var}\left\{\bW\br\right\}
\end{array}
\end{equation}

In this way, ${\bbf}_{BLU}$ will be unbiased and efficient (within
the class of linear estimators) by design. The resulting best
linear unbiased estimator is (Gauss-Markov Theorem):

\begin{equation}
\begin{array}{rcl}
{\bbf}_{\rm BLU}$=$(\bcS^{\rm
T}\bC_{\bar{\bn}}^{-1}\bcS)^{-1}\bcS^{\rm
T}\bC_{\bar{\bn}}^{-1}\br
\end{array} \label{BLUE}
\end{equation}

\noindent with the covariance matrix of ${\bbf}_{BLU}$ is

\begin{equation}
\begin{array}{rcl}
{\bC}_{\bbf_{\rm BLU}}$=$(\bcS^{\rm
T}\bC_{\bar{\bn}}^{-1}\bcS)^{-1}
\end{array}.
\end{equation}

\noindent Since the above data are of Gaussian distributions, the
BLUE in (\ref{BLUE}) is also the minimum variance unbiased
estimation.

The linear filter representation of this detector is

\begin{equation}
\begin{array}{rcl}
{\bw}_{\rm BLU}$=$\bg\bC_{\bar{\bn}}^{-T}\bcS(\bcS^{\rm
T}\bC_{\bar{\bn}}^{-1}\bcS)^{-1}\
\end{array}. \label{w_BLUE}
\end{equation}



\subsection{Minimum Mean-Squared Estimation}
Now we propose a linear estimator based on MMSE criterion. This
class of estimators are generically termed Wiener filter. Given
measurements $\br$, the MSE estimator of $\bbf$, ${\bbf}_{\rm MS}
= f( \br )$, minimizes the mean-squared error $J_{\rm
MS}=E\{||\bbf-\hat{\bbf}||_2^2\}$. The function $f(\br)$ may be
nonlinear or linear and its exact structure is determined by
minimizing $J_{\rm MS}$. When $\bbf$ and $\br$ are jointly
Gaussian, the linear estimator that minimizes the mean-sqared
error is (Bayesian Gauss-Markov Theorem)

\begin{equation}
\begin{array}{rcl}
{\bbf}_{\rm MS} &=& (\bC_{\bbf}^{-1}+\bcS^{\rm
T}\bC_{\bar{\bn}}^{-1}\bcS)^{-1}\bcS^{\rm
T}\bC_{\bar{\bn}}^{-1}\br
\end{array} \label{MSE}
\end{equation}

\noindent with the variance matrix of ${\bbf}_{\rm MS}$ is

\begin{equation}
\begin{array}{rcl}
\bC_{{\bbf_{\rm MS}}} &=& (\bC_{\bbf}^{-1}+\bcS^{\rm
T}\bC_{\bar{\bn}}^{-1}\bcS)^{-1}
\end{array}.
\end{equation}

The linear filter representation of this blind MMSE detector is

\begin{equation}
\begin{array}{rcl}
{\bw}_{\rm MS} &=&\bg
\bC_{\bar{\bn}}^{-T}\bcS(\bC_{\bbf}^{-1}+\bcS^{\rm
T}\bC_{\bar{\bn}}^{-1}\bcS)^{-1}
\end{array} \label{w_MSE}
\end{equation}


\subsection{Kalman Filtering Estimation}
Kalman filter can be taken as an important generalization of
Wiener filter with the ability to accommodate vector signals and
noise which additionally may be nonstationary. It may also be
thought of as a sequential MMSE estimator of a signal embedded in
noise and this signal can be characterized by a state model. If
the signal and noise are jointly Gaussian, then the Kalman filter
is an optimal MMSE estimator, and if not, it is the optimal linear
MMSE estimator. To develop a blind adaptive multiuser detector
based on Kalman filtering in~(\ref{KFE_SF_X}) and
(\ref{KFE_SF_Y}), a linear first-order state space model have to
be devised.


\begin{equation}
\begin{array}{rcl}
\mathbf{x}[n]&=&\mathbf{P}\mathbf{x}[n-1] +
\mathbf{Q}\mathbf{u}[n]
\end{array} \label{KFE_SF_X}
\end{equation}

\begin{equation}
\begin{array}{rcl}
\mathbf{y}[n]&=&\mathbf{H}\mathbf{x}[n] + \mathbf{w}[n]
\end{array} \label{KFE_SF_Y}
\end{equation}




\bibliographystyle{unsrt}
\bibliography{FastBDD}

\end{document}

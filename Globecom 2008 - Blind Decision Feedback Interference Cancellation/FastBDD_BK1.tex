
\documentclass[a4paper,11pt,fleqn]{article}
\usepackage{amsfonts}
\usepackage{amsthm}
\usepackage{graphicx}

\setlength{\parindent}{3em} \setlength{\oddsidemargin}{0in}
\setlength{\textwidth}{6.5in} % sets 1in left and right margins
\setlength{\topmargin}{0.0in} % change to 0.2in for regular latex
%\setlength{\headheight}{0in}
%\setlength{\footheight}{0.5in}
\setlength{\footskip}{0.5in}
\setlength{\textheight}{9.0in} %sets 1in top and bottom margins
%\renewcommand{\baselinestretch}{1} %set to 1.5 for double spacing.


\newtheorem{Prop}{Proposition}
\newtheorem{lemma}{Lemma}

\newcommand{\br}{{\mathbf r}}
\newcommand{\bA}{{\mathbf A}}
\newcommand{\ba}{{\bf a}}
\newcommand{\bb}{{\bf b}}
\newcommand{\bc}{{\bf c}}
\newcommand{\bC}{{\bf C}}
\newcommand{\bG}{{\bf G}}
\newcommand{\bd}{{\bf d}}
\newcommand{\be}{{\bf e}}
\newcommand{\bs}{{\bf s}}
\newcommand{\bn}{{\bf n}}
\newcommand{\bu}{{\bf u}}
\newcommand{\bv}{{\bf v}}
\newcommand{\bw}{{\bf w}}
\newcommand{\bx}{{\bf x}}
\newcommand{\bbf}{{\bf f}}
\newcommand{\bF}{{\bf F}}
\newcommand{\bL}{{\bf L}}
\newcommand{\bM}{{\bf M}}
\newcommand{\bN}{{\bf N}}
\newcommand{\bS}{{\bf S}}
\newcommand{\bT}{{\bf T}}
\newcommand{\bD}{{\bf D}}
\newcommand{\bX}{{\bf X}}
\newcommand{\bP}{{\bf P}}
\newcommand{\bI}{{\bf I}}
\newcommand{\bR}{{\bf R}}
\newcommand{\bU}{{\bf U}}
\newcommand{\bV}{{\bf V}}
\newcommand{\bW}{{\bf W}}
\newcommand{\bJ}{{\bf J}}
\newcommand{\bB}{{\bf B}}
\newcommand{\bzero}{{\bf 0}}
\newcommand{\bgamma}{{\mbox {\boldmath $\gamma$}}}
\newcommand{\btheta}{{\mbox {\boldmath $\theta$}}}
\newcommand{\bLambda}{{\mbox {\boldmath $\Lambda$}}}
\newcommand{\bPsi}{{\mbox {\boldmath $\Psi$}}}
\newcommand{\bPhi}{{\mbox {\boldmath $\Phi$}}}
\newcommand{\bcS}{{\mbox {\boldmath ${\cal S}$}}}
\newcommand{\bcH}{{\mbox {\boldmath ${\cal H}$}}}
\newcommand{\bcI}{{\mbox {\boldmath ${\cal I}$}}}
\newcommand{\bcR}{{\mbox {\boldmath ${\cal R}$}}}
\newcommand{\bcB}{{\mbox {\boldmath ${\cal B}$}}}

\title{Fast Multiuser Blind Decorrelating Detection Algorithms}
\date{}
\author{}
\begin{document}
\maketitle

\begin{abstract}
Multiuser detection is one of the key techniques for combating
multiple access interference (MAI) in CDMA systems.
\end{abstract}

\section{Introduction}

Direct-sequence code division multiple access (DS/CDMA) techniques
have attracted increasing attention for efficient use of available
bandwidth, resistance to interference and flexibility to variable
traffic patterns. Multiuser detection strategy is a method to
minimize the effect of MAI and solve the near-far problem in CMDA
systems. It has been extensively investigated over the past
several years~\cite{Verd98}, since MAI is the dominant impairment
for CDMA systems and exists even in perfect power-controlled CDMA
systems. Most early work on multiuser detection assumed that the
receiver knew the spreading codes or had some knowledge of all
users, then exploited this knowledge to combat MAI. For example,
the classic decorrelating detector could achieve the optimum
near-far resistance and completely eliminate MAI from other users
with the expense of enhancement of background noise. However, in
many practical cases, especially in a dynamic environment, e.g. in
the downlink of the CDMA system, it is very difficult for a mobile
user to obtain accurate information on other active users in the
same channel. On the other hand, the frequent use of training
sequence is certainly a waste of channel bandwidth. So multiuser
blind detection has been proposed. Recent research has been
devoted to the blind multiuser receivers and subspace-based
signature waveform estimation schemes to achieve better
performance and higher capacity~\cite{Honi95, Poor97, Wang98,
Torl97, Liu96}. The Minimum output energy (MOE) method and
subspace method were presented for multiuser blind detection with
the knowledge of only the desired users' spreading code and
possible timing.

The large gaps in performance and complexity between the
conventional single-user matched filter and optimum multiuser
detector encourage the search for other multiuser detectors that
exhibit good performance/complexity tradeoffs. The classic
decorrelating detector~\cite{Lupa89} not only is a simple and
natural strategy but it is optimal according to three different
criteria: least squares, near-far resistance~\cite{Verd86} and
maximum-likelihood when the received amplitudes are
unknown~\cite{Lupa89}. Though it does not require knowledge of the
received amplitudes, it does require some information about other
active users' signature waveforms. This makes it difficult be put
in practical applications.

Multiuser blind detection using subspace techniques was first
developed in depth by Wang and Poor~\cite{Wang98, Poor98}. Such
techniques were appropriate for the downlink environment where
only the desired users' code is available. More recently, these
subspace techniques were extended by Wang and
Host-Madsen~\cite{Wang99}, named group multiuser blind detectors,
to uplink environments where the base station knows the codes of
in-cell users, but not those of users outside the cell. In the
subspace-based blind detection approach~\cite{Wang98}, the linear
detectors are constructed in the closed form once the signal
subspace components are computed and offer lower computational
complexity and better performance than the blind MOE detector. For
the subspace-based blind adaptive detector, the project
approximation subspace tracking deflation (PASTd)
algorithm~\cite{Yang95} is used to estimate the signal subspace.

It is shown in~\cite{Honi95} that the MOE receiver is equivalent
to the linear minimum mean square error (MMSE) detector, which is
near-far resistant and has much less complexity compared the
optimal multiuser detection. The major limitation of MOE schemes
to multiuser blind detection is that there is a satuaration effect
in the steady state, which causes a significant performance gap
between the converged blind MOE and the true MMSE
detector~\cite{Honi95}.


\section{Data Model and Problem Description}

The basic CDMA $K$-user channel model, consisting of the sum of
antipodally modulated synchronous signature waveforms embedded in
additive white Gaussian noise (AWGN), is considered here. The
received base-band signal during one symbol interval in such a
channel can be modeled as:

\begin{equation}
\matrix{r(t)&=&\sum\limits_{k=1}^{K}A_k b_k^{(n)} s_k (t)+ n(t)}
\end{equation}

\noindent where $t\in [(n-1)T,\ nT]$, $T$ is the symbol interval.
$n(t)$ represents the Gaussian channel noise, $K$ is the number of
users and $A_k$, $b_k^{(n)}$ denote the received amplitude and
data bit of the $k$th user, respectively. It is assumed that
$b_k^{(n)}\in\{-1,\ +1\}$ is a collection of independent
equiprobable $\pm1$ random variables transmitted by the $k$th user
during $[(n-1)T,\ nT]$ and $s_k(t)$ denotes the normalized signal
waveform of the $k$th user on the interval $[(n-1)T,\ nT]$, i.e.,
$\|s_k(t)\|=1$.

The received signal $r(t)$ is passed through a chip-matched filter
followed by a chip-rate sampler. As a result, $r(t)$, $t\in
[(n-1)T,\ nT]$, is converted into a $L\times 1$ column vector
$\br$ of the samples of the chip-matched filter outputs within a
symbol interval $[(n-1)T,\ nT]$ as

\begin{equation}
\begin{array}{rcl}
\br &=& \left[\matrix{r_n^1 & r_n^2 & \ldots & r_n^L}\right]^T\\
 &=& \sum\limits_{k=1}^{K} A_k b_k^{(n)} \bs_k + \bn \\
 &=& \bS \bA \bb + \bn
\end{array} \label{r}
\end{equation}

\noindent where $\bA$ is the received amplitude diagonal matrix,
$\bS = [\bs_1\ \bs_2\ \ldots\ \bs_K]$ is the $L \times K$
signature matrix with the $k$th column $\bs_k$ being the signature
vector of the $k$th user, $\bb = [b_1^{(n)}\ b_2^{(n)}\ \ldots\
b_K^{(n)}]^T = [b_1^{(n)}\ \tilde{\bb}^T]^T$ is the information
vector sent by all the $K$ users at time $t=n$ and $b_1^{(n)}$ is
the bit sent by the first user at time $t=n$, and $\bn$ is an
$L$-dimensional Gaussian vector with independent
$\sigma^2$-variance components. We maintain the restriction that
$L \geq K$.

Most of the linear multiuser detectors for demodulating the $k$th
user's data bit in (\ref{r}) is in the form of a correlator
followed by a hard limiter, which could be expressed as

\begin{equation}
\begin{array}{rcl}
\hat{b}_k^{(n)} &=& \mbox{sign}\{\bw_k^T\br\}
\end{array} \label{linear}
\end{equation}

\noindent where $\bw_k \in \mathbb{R}^{L\times 1}$ is the linear
representation of multiuser detector.

Linear multiuser detectors can be implemented in a decentralized
fashion where only the user or users of interest need be
demodulated.



\section{New Data Model with Semi-Blind Signature Matrix}

Most of the multiuser detection schemes assume the knowledge of
the spreading codes and/or channel parameters of all the users
that contribute to the received signal. These receivers then
exploit this knowledge to combat MAI. A second form of detectors,
know as the blind detectors, operate without knowledge of the
channel input and the information of other users. Usually, many
practical systems lie in between these two extremes. The first
form of detectors are too optimistic as we can not expect to know
the signature waveforms of all the users, while the latter
under-utilizes the our knowledge of the system. In this section,
we develop two one-shot multiuser semi-blind decorrelating
detectors that are expected to bridge the performance gap between
the conventional detectors and blind ones.


Without loss of the generality, assume only the bits $b_1^{(n)}$
sent by the first user is considered in the following description.
And a new $L\times M$ semi-blind signature matrix $\bcS$ could be
defined as

\begin{equation}
\begin{array}{rcl}
\bcS&=&[\matrix{\bar{\bs}_1&\bar{\bs}_2&\bar{\bs}_3&\ldots&\bar{\bs}_M}]\\
 &=&[\matrix{A_1\bs_1&\bar{\br}_{1}&\bar{\br}_{2}&\ldots&\bar{\br}_{M-1}}]\\
 &=&\left[\matrix{\bS\bA\be_1&\bS\bA\bar{\bb}_1&\bS\bA\bar{\bb}_2&\ldots&\bS\bA\bar{\bb}_{M-1}}\right]+ \bar{\bN}\\
 &=&\bS\bA\left[\matrix{\be_1&\bar{\bb}_1&\bar{\bb}_2&\ldots&\bar{\bb}_{M-1}}\right]+ \bar{\bN}\\
 &=&\bS\bA\left[\matrix{\be_1 & \bD }\right]+ \bar{\bN}\\
 &=&\bS\bA\bB + \bar{\bN}
\end{array} \label{S}
\end{equation}

\noindent where $L\geq M\geq K$, $\bar{\br}_i$, $i=1,\ 2,\
\ldots,\ M-1$, are the arbitrary received vectors which are
linearly independent to each other and the $K\times 1$ vector
$\bar{\bb}_m$ is the corresponding information vector which
denotes the information sent by all $K$ users. $\bD = [\bar{\bd}\
\tilde{\bD}^T]^T$, The $(K-1)\times 1 $ vector $\bar{\bd}$ is the
information vector consist of the known bits sent by the current
user previously. $\mbox{rank}\{\tilde{\bD}\}=K-1$,
$\bar{\bN}=[\mathbf{0}\ \tilde{\bN}]$ and

\begin{equation}
\begin{array}{rcl}
 \bB&=&\left[\matrix{\be_1 & \bD }\right]\\
 &=&\left[\matrix{\be_1 & \matrix{ \bar{\bd}^T\cr\tilde{\bD}} }\right]\\
 &=&\left[\matrix{\bc^{T} \cr \matrix{\mathbf{0}& \tilde{\bD}}
 }\right]\\
 &=&\left[\matrix{1& \bd^{T} \cr \mathbf{0}& \tilde{\bD} }\right]

\end{array}
\end{equation}

\noindent and $\mbox{rank}\{\bB\}=K$.

As in equation (\ref{r}) and (\ref{S}), the relationship between
the received signal vector $\br$ and the new blind signature
matrix $\bcS$ can be expressed as

\begin{equation}
\begin{array}{rcl}
\br&=&\bS\bA\bb + \bn\\
 &=&\bS\bA\bB\bB^{+}\bb + \bn\\
 &=&(\bcS-\bar{\bN})\bB^{+}\bb+ \bn\\
 &=&\bcS\bB^{+}\bb -\bar{\bN}\bB^{+}\bb + \bn\\
 &=&\bcS\bd + \tilde{\bn} \label{rn}
\end{array}
\end{equation}

\noindent where $\bB^{+} = \bB^T(\bB\bB^T)^{-1} $ is Moor-Penrose
general inverse of $\bB$ and $\bd$ denotes the new $K \times 1$
detection vector and is defined as

\begin{equation}
\begin{array}{rcl}
\bd&=&\left[\matrix{\be_1&\bD}\right]^{+}\bb\\
 &=&\left[\matrix{1&\bar{\bd}^T\cr\mathbf{0}&\tilde{\bD}}\right]^{+}\left[\matrix{b_1^{(n)}\cr\tilde{\bb}}\right]
\end{array} \label{DetectorVector}
\end{equation}

with covariance matrix $\bR_{\bd}$ given by

\begin{equation}
\begin{array}{rcl}
\bR_{\bd}&=&E\{\bd\bd^{H}\}\\
&=&E\{(\bB^{+}\bb)(\bB^{+}\bb)^{H}\}\\
&=&\bB^{+}\bB^{+H}\\
&=&\bR_{\bB}^{+}
\end{array}
\end{equation}


\noindent and $\tilde{\bn}$ is the new noise vector and defined as

\begin{equation}
\begin{array}{rcl}
\tilde{\bn}&=&\bn-\bar{\bN}\bB^{+}\bb
\end{array} \label{new_noise}
\end{equation}

\noindent with covariance matrix $\bR_{\tilde{\bn}}$ defined by

\begin{equation}
\begin{array}{rcl}
\bR_{\tilde{\bn}}&=&E\{(\bn-\bar{\bN}\bB^{+}\bb)(\bn-\bar{\bN}\bB^{+}\bb)^H\}\\
&=& \sigma^{2}\bI+\bar{\bN}\bB^{+}\bB^{+H}\bar{\bN}^{H}
\end{array}
\end{equation}

Furthermore, the following result could be easily proved.

\begin{lemma}
The inverse of $\bB$ is
\begin{equation}
\begin{array}{rcl}
\bB^{+}&=&\left[\matrix{\be_1&\matrix{-\bar{\bd}^T\tilde{\bD}^+\cr\tilde{\bD}^+}}\right]
\end{array}
\end{equation}
\end{lemma}

\begin{proof}

Based on the definition of $\bB$,

\begin{equation}
\begin{array}{rcl}
\bB&=&\left[\be_1\ \matrix{\bar{\bd}^T\cr\tilde{\bD}}\right]
\end{array}
\end{equation}

\noindent so that,

\begin{equation}
\left[\be_1\ \matrix{\bar{\bd}^T\cr\tilde{\bD}}\right]\left[\be_1\
\matrix{-\bar{\bd}^T\tilde{\bD}^+\cr\tilde{\bD}^+}\right]
=\left[\matrix{1&\bar{\bd}^T\cr\mathbf{0}&\tilde{\bD}}\right]\left[\matrix{1&-\bar{\bd}^T\tilde{\bD}^+\cr\mathbf{0}&\tilde{\bD}^+}\right]
=\bI
\end{equation}
\end{proof}

So, with the above lemma, the detection vector $\bd$ can be
re-written as

\begin{equation}
\begin{array}{rcl}
\bd&=&\left[\matrix{d^1\cr\tilde{\bd}}\right]\\
 &=&\left[\matrix{\be_1&\matrix{\bar{\bd}^T\cr\tilde{\bD}}}\right]^+\left[\matrix{ b_1^{(n)} \cr \tilde{\bb} }\right]\\
 &=&\left[\matrix{\be_1&\matrix{-\bar{\bd}^T\tilde{\bD}^+\cr\tilde{\bD}^+}}\right]\left[\matrix{b_1^{(n)}\cr\tilde{\bb}}\right]\\
 &=&\left[\matrix{b_1^{(n)}-\bar{\bd}^T\tilde{\bD}^+\tilde{\bb}\cr\tilde{\bD}^+\tilde{\bb}}\right]
\end{array}
\end{equation}

\noindent where $d^1$ is the first element of the detection vector
$\bd$ as

\begin{equation}
\matrix{d^{1}$=$b_1^{(n)}-\bar{\bd}^T\tilde{\bD}^+\tilde{\bb}}
\end{equation}

\noindent and $\tilde{\bd}$ is the corresponding complement vector
as

\begin{equation}
\matrix{\tilde{\bd}$=$\tilde{\bD}^+\tilde{\bb}}
\end{equation}


Now, the following result could be very easily researched.

\begin{lemma}
The bit $b_1^{(n)}$ sent by the first user at time $t=n$ can be
got with the following equation.
\begin{equation}
\begin{array}{rcl}
b_1^{(n)} &=&d_{n}^{1}+\bar{\bd}^T\tilde{\bd}\\
 &=& [\matrix{1&\bar{\bd}^T}]\bd\\
 &=& \bc^T\bd
\end{array}
\end{equation} \label{bn_estimation}
\end{lemma}

By now, as we may see, after the definition of the new blind
signature matrix $\bcS$ in equation (\ref{S}), the form of the
classic multiuser detection model is still kept as in equation
(\ref{rn}). But the different is that, the information bit vector
$\bb$ is replaced by the detection vector $\bd$ as in equation
(\ref{DetectorVector}) and the original AWGN vector $\bn$ is
replace by the new noise vector $\tilde{\bn}$ as in equation
(\ref{new_noise}). Fortunately, on the other hand, with lemma
\ref{bn_estimation}, it still is possible and easy for us to
calculate the bit vector $\bb$ with the detection vector $\bd$ and
the previously detected bits vector $\bc$. So, the final question
is how to estimate the detection vector $\bd$ as efficiently as
possible.

\section{Minimum Variance Unbiased Estimation}
We know that when the data observed can be modelled as (\ref{rn}),
$\bcS$ is a known determined observation matrix and $\tilde{\bn}$
is white gaussian noise with PDF ${\cal N}(0, \sigma^{2}\bI)$,
then the minimum variance unbiased (MVU) estimation can be written
by
\begin{equation}
\begin{array}{rcl}
\hat{\bd} &=& (\bcS^T\bcS)^{-1}\bcS^T\br
\end{array}.\label{MVU0}
\end{equation}

\noindent with the covariance matrix of $\hat{\bd}$ is
\begin{equation}
\begin{array}{rcl}
\bR_{\bd} &=& \sigma^{2}(\bcS^T\bcS)^{-1}
\end{array}.
\end{equation}

For the above assumptions, the MVU estimation (\ref{MVU0}) is
efficient in that it attains the CRLB. However, when the noise
$\tilde{\bn}$ is not white with PDF ${\cal N}(0,
\bR_{\tilde{\bn}})$, the MVU estimation of $\bd$ is given by

\begin{equation}
\begin{array}{rcl}
\hat{\bd}
&=&(\bcS^T\bR_{\tilde{\bn}}^{-1}\bcS)^{-1}\bcS^T\bR_{\tilde{\bn}}^{-1}\br
\end{array}.
\end{equation}

\section{ Best Linear Unbiased Estimation}
We begin by assuming the following linear structure
$\hat{\bd}_{BLU}=\bG\br$ for this so-called best linear unbiased
estimator (BLUE). Matrix $\bG$ is designed such that: (1) $\bcS$
must be deterministic, and (2) $\tilde{\bn}$ must be zero mean
with positive definite known covariance matrix
$\bR_{\tilde{\bn}}$. (3) $\hat{\bd}_{BLU}$ is an unbiased
estimator of $\bd$, and (4) the error variance for each of the n
parameters is minimized. In this way, $\hat{\bd}_{BLU}$ will be
unbiased and efficient (within the class of linear estimators) by
design. The resulting best linear unbiased estimator is:

\begin{equation}
\matrix{ \hat{\bd}_{BLU} $=$
(\bcS^H\bR_{\tilde{\bn}}^{-1}\bcS)^{-1}\bcS^H\bR_{\tilde{\bn}}^{-1}\br
}
\end{equation}

\section{ Least-Squares Estimation }
At first, we assume the measurements of $\bcS$ is assumed to be
free of error. Hence, all errors are confined to the received
vector $\br$. Then the following classic LS estimation is
proposed.

\begin{lemma}~\cite{Golu96} Suppose $\bU^T\bcS\bV=\mathbf{\Sigma}$ is the SVD of $\bcS\in\mathbb{R}^{L\times
 K}$ with $r=rank(\bcS)$. And if $\bU=[\matrix{\bu_1&\bu_2&\ldots&\bu_L}]$,
 $\bV=[\matrix{\bv_1&\bv_2&\ldots&\bv_K}]$, $\mathbf{\Sigma}=diag\{[\matrix{\sigma_1&\ldots\sigma_r&0&\ldots&0}]\}$ and $\br\in \mathbb{R}^{L\times 1}$, then

 \begin{equation}
 \matrix{\bd_{LS}&=&\sum\limits_{i=1}^{r}\frac{\bu_i^T\br}{\sigma_i}\bv_i&=&\bcS^+\br}
 \end{equation}

\noindent minimizes $\|\bcS\bd-\br\|_2$ and has the smallest
2-norm of all minimizers. Moreover
 \begin{equation}
 \matrix{\varepsilon_{LS}^2 &=& \min\limits_{\bx\in\mathbb{R}}\|\bcS\bx-\br\|_2^2 &=& \sum\limits_{i=r+1}^{L}(\bu_i^T\br)^2}
 \end{equation}
\label{VectorLS}
\end{lemma}

\begin{proof}
For any $\bx\in\mathbb{R}^{K\times 1}$, we have

\begin{equation}
\begin{array}{rcl}
\|\bcS\bx-\br\|_2^2&=&\|(\bU^T\bcS\bV)(\bV^T\bx)-\bU^T\br\|_2^2\\
        &=& \| \mathbf{\Sigma} \mathbf{\alpha} - \bU^T\br \|_2^2\\
        &=& \sum\limits_{i=1}^{r}(\sigma_i\alpha_i-\bu_i^T\br) + \sum\limits_{i=r+1}^{m}(\bu_i^T\br)^2
\end{array}
\end{equation}

\noindent where $\mathbf{\alpha} = \bV^T\bx$. Clearly, if $\bx$
solves the LS problem, then $\alpha_i = \bu_i^T\br/\sigma_i$ for
$i=1,\ 2,\ \ldots,\ r$. If we set
$\alpha_{r+1}=\alpha_{r+2}=\ldots=\alpha_{n}$, then the resulting
$\bx=\bd_{LS}$ clearly has minimal 2-norm.
\end{proof}

So, the least squares estimation of $\bd$ is

\begin{equation}
\begin{array}{rcl}
\bd_{LS} &=& \bcS^+\br\\
 &=&\bd + \bcS^+\tilde{\bn}
\end{array} \label{LS}
\end{equation}

\noindent where the $K\times L$ matrix $\bcS^+$ is the
Moore-Penrose generalized inverse of $\bcS$.

\section{Minimum Mean-Squared Estimation}
Given measurements $\br$, the mean-sqaured estimator (MSE) of
$\bd$, $\hat{\bd}_{MS} = f( \br )$, minimizes the mean-squared
error $J_{MSE}=E\{||\bd-\hat{\bd}||_2^2\}$. The function $f(\br)$
may be nonlinear or linear and its exact structure is determined
by minimizing $J_{\bd}$. When $\bd$ and $\br$ are jointly
Gaussian, the estimator that minimizes the mean-sqared error is
(Bayesian Gauss-Markov Theorem)

\begin{equation}
\begin{array}{rcl}
\hat{\bd}_{MS} &=&
(\bR_{\bd}^{-1}+\bcS^{H}\bR_{\tilde{\bn}}^{-1}\bcS)^{-1}\bcS^{H}\bR_{\tilde{\bn}}^{-1}\br
\end{array} \label{MSE}
\end{equation}





\bibliographystyle{unsrt}
\bibliography{FastBDD}

\end{document}

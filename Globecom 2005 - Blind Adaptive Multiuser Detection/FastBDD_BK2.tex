
\documentclass[a4paper,11pt,fleqn]{article}
\usepackage{amsfonts}
\usepackage{amsthm}
\usepackage{graphicx}

\setlength{\parindent}{3em} \setlength{\oddsidemargin}{0in}
\setlength{\textwidth}{6.5in} % sets 1in left and right margins
\setlength{\topmargin}{0.0in} % change to 0.2in for regular latex
%\setlength{\headheight}{0in}
%\setlength{\footheight}{0.5in}
\setlength{\footskip}{0.5in}
\setlength{\textheight}{9.0in} %sets 1in top and bottom margins
%\renewcommand{\baselinestretch}{1} %set to 1.5 for double spacing.


\newtheorem{Prop}{Proposition}
\newtheorem{lemma}{Lemma}

\newcommand{\br}{{\mathbf r}}
\newcommand{\bA}{{\mathbf A}}
\newcommand{\ba}{{\bf a}}
\newcommand{\bb}{{\bf b}}
\newcommand{\bc}{{\bf c}}
\newcommand{\bC}{{\bf C}}
\newcommand{\bG}{{\bf G}}
\newcommand{\bd}{{\bf d}}
\newcommand{\be}{{\bf e}}
\newcommand{\bs}{{\bf s}}
\newcommand{\bm}{{\bf m}}
\newcommand{\bn}{{\bf n}}
\newcommand{\bu}{{\bf u}}
\newcommand{\bv}{{\bf v}}
\newcommand{\bw}{{\bf w}}
\newcommand{\bx}{{\bf x}}
\newcommand{\bbf}{{\bf f}}
\newcommand{\bF}{{\bf F}}
\newcommand{\bL}{{\bf L}}
\newcommand{\bM}{{\bf M}}
\newcommand{\bN}{{\bf N}}
\newcommand{\bS}{{\bf S}}
\newcommand{\bT}{{\bf T}}
\newcommand{\bD}{{\bf D}}
\newcommand{\bX}{{\bf X}}
\newcommand{\bP}{{\bf P}}
\newcommand{\bI}{{\bf I}}
\newcommand{\bR}{{\bf R}}
\newcommand{\bU}{{\bf U}}
\newcommand{\bV}{{\bf V}}
\newcommand{\bW}{{\bf W}}
\newcommand{\bJ}{{\bf J}}
\newcommand{\bB}{{\bf B}}
\newcommand{\bzero}{{\bf 0}}
\newcommand{\bgamma}{{\mbox {\boldmath $\gamma$}}}
\newcommand{\btheta}{{\mbox {\boldmath $\theta$}}}
\newcommand{\bLambda}{{\mbox {\boldmath $\Lambda$}}}
\newcommand{\bPsi}{{\mbox {\boldmath $\Psi$}}}
\newcommand{\bPhi}{{\mbox {\boldmath $\Phi$}}}
\newcommand{\bcS}{{\mbox {\boldmath ${\cal S}$}}}
\newcommand{\bcH}{{\mbox {\boldmath ${\cal H}$}}}
\newcommand{\bcI}{{\mbox {\boldmath ${\cal I}$}}}
\newcommand{\bcR}{{\mbox {\boldmath ${\cal R}$}}}
\newcommand{\bcB}{{\mbox {\boldmath ${\cal B}$}}}

\title{Blind Multiuser Detection Algorithms}
\date{}
\author{}
\begin{document}
\maketitle

\begin{abstract}
Multiuser detection is one of the key techniques for combating
multiple access interference (MAI) in CDMA systems. In this paper,
we propose a new blind multiuser model using a blind spreading
sequence matrix and several blind multiuser detection algorithms
using least-squares (LS) estimations, best least unbiased (BLU)
estimation, minimum mean-squared error (MMSE) estimation and
Kalman estimation criteria. Theoretical analysis and computer
simulations are provided to demonstrate the proposed schemes too.
\end{abstract}

\section{Introduction}

Direct-sequence code division multiple access (DS/CDMA) techniques
have attracted increasing attention for efficient use of available
bandwidth, resistance to interference and flexibility to variable
traffic patterns. Multiuser detection strategy is a method to
minimize the effect of MAI and solve the near-far problem in CMDA
systems. It has been extensively investigated over the past
several years~\cite{Verd98}, since MAI is the dominant impairment
for CDMA systems and exists even in perfect power-controlled CDMA
systems. Most early work on multiuser detection assumed that the
receiver knew the spreading codes or had some knowledge of all
users, then exploited this knowledge to combat MAI. For example,
the classic decorrelating detector could achieve the optimum
near-far resistance and completely eliminate MAI from other users
with the expense of enhancement of background noise. However, in
many practical cases, especially in a dynamic environment, e.g. in
the downlink of the CDMA system, it is very difficult for a mobile
user to obtain accurate information on other active users in the
same channel. On the other hand, the frequent use of training
sequence is certainly a waste of channel bandwidth. So multiuser
blind detection has been proposed. Recent research has been
devoted to the blind multiuser receivers and subspace-based
signature waveform estimation schemes to achieve better
performance and higher capacity~\cite{Honi95, Poor97, Wang98,
Torl97, Liu96}. The Minimum output energy (MOE) method and
subspace method were presented for multiuser blind detection with
the knowledge of only the desired users' spreading code and
possible timing.

The large gaps in performance and complexity between the
conventional single-user matched filter and optimum multiuser
detector encourage the search for other multiuser detectors that
exhibit good performance/complexity tradeoffs. The classic
decorrelating detector~\cite{Lupa89} not only is a simple and
natural strategy but it is optimal according to three different
criteria: least squares, near-far resistance~\cite{Verd86} and
maximum-likelihood when the received amplitudes are
unknown~\cite{Lupa89}. Though it does not require knowledge of the
received amplitudes, it does require some information about other
active users' signature waveforms. This makes it difficult be put
in practical applications.

Multiuser blind detection using subspace techniques was first
developed in depth by Wang and Poor~\cite{Wang98, Poor98}. Such
techniques were appropriate for the downlink environment where
only the desired users' code is available. More recently, these
subspace techniques were extended by Wang and
Host-Madsen~\cite{Wang99}, named group multiuser blind detectors,
to uplink environments where the base station knows the codes of
in-cell users, but not those of users outside the cell. In the
subspace-based blind detection approach~\cite{Wang98}, the linear
detectors are constructed in the closed form once the signal
subspace components are computed and offer lower computational
complexity and better performance than the blind MOE detector. For
the subspace-based blind adaptive detector, the project
approximation subspace tracking deflation (PASTd)
algorithm~\cite{Yang95} is used to estimate the signal subspace.

It is shown in~\cite{Honi95} that the MOE receiver is equivalent
to the linear MMSE detector, which is near-far resistant and has
much less complexity compared the optimal multiuser detection. The
major limitation of MOE schemes to multiuser blind detection is
that there is a satuaration effect in the steady state, which
causes a significant performance gap between the converged blind
MOE and the true MMSE detector~\cite{Honi95}.

As we see, various multiuser detection schemes have been developed
to mitigate the effects of MAI. Blind detectors are obviously more
close to the practical application. However, since only the
signature information of the desired user could be available,
nearly all the proposed blind detectors need some adaptive or
search procedure to collection information of other users and/or
channel before detection. In this work, we consider another
different idea to develop blind multiuser detectors. Instead of
searching for the original spreading sequences, a blind spreading
matrix consisting of the desired user's original spreading
sequence vector and some previous received signal vectors is
constructed. We show that the bits sent for desired user can be
clearly detected if there is no noise or distortion in the
proposed blind spreading sequence matrix. In practical situations,
when the proposed spreading matrix is corrupted by channels, we
proposed many different estimation schemes for blind multiuser
detection, which include LS-liked estimations, BLU estimation,
linear MMSE estimation and Kalman filter estimation. In the
present multiuser blind detectors, only the signature and timing
of the desired user are utilized. There are no any adaptive or
search procedures employed as in other multiuser blind detection
algorithms. Theoretical analysis and computer simulations are also
presented to demonstrate the performance of these blind detectors.

The rest of the paper is organized as follows. In Section II, we
summarize the signal model. In Section III, we propose a new blind
multiuser model and present the blind multiuser detection
framework. In Section IV, various estimation schemes are discussed
for blind detection. Performance analysis and simulation results
are provided in Section V and VI. Section VII concludes this
papers.


\section{Classic Multiuser Channel Model And Problem Description}

The basic CDMA $K$-user channel model, consisting of the sum of
antipodally modulated synchronous signature waveforms embedded in
additive white Gaussian noise (AWGN), is considered here. The
received base-band signal during one symbol interval in such a
channel can be modelled as:

\begin{equation}
\matrix{r(t)&=&\sum\limits_{k=1}^{K}A_k b_k [n]s_k (t)+ n(t)}
\end{equation}

\noindent where $t\in [nT,\ (n+1)T]$, $T$ is the symbol interval.
$n(t)$ represents the Gaussian channel noise, $K$ is the number of
users and $A_k$, $b_k[n]$ denote the received amplitude and data
bit of the $k$th user, respectively. It is assumed that
$b_k[n]\in\{-1,\ +1\}$ is a collection of independent equiprobable
$\pm1$ random variables transmitted by the $k$th user during
$[nT,\ (n-1)T]$ and $s_k(t)$ denotes the normalized signal
waveform of the $k$th user on the interval $[(n-1)T,\ nT]$, i.e.,
$\|s_k(t)\|=1$. The received signal $r(t)$ is passed through a
chip-matched filter followed by a chip-rate sampler. As a result,
$r(t)$, $t\in [(n-1)T,\ nT]$, is converted into a $L\times 1$
column vector $\br$~\footnote{Without the loss of generality, we
drop parameter $n$ and denote $\br=\br[n]$. The same to $b_k$} of
the samples of the chip-matched filter outputs within a symbol
interval $[(n-1)T,\ nT]$ as

\begin{equation}
\begin{array}{rcl}
\br &=& \left[\matrix{r_1 & r_2 & \ldots & r_L}\right]^T\\
 &=& \sum\limits_{k=1}^{K} A_k b_k \bs_k + \bn \\
 &=& \bS \bA \bb + \bn
\end{array} \label{r}
\end{equation}

\noindent where $\bA=\mbox{diag}\{[\matrix{A_1\ A_2\ \ldots\
A_K}]\}$ is the received amplitude diagonal matrix, $\bS = [\bs_1\
\bs_2\ \ldots\ \bs_K]$ is the $L \times K$ signature matrix with
the $k$th column $\bs_k$ being the signature vector of the $k$th
user, $\bb = [b_1\ b_2\ \ldots\ b_K]^T = [b_1\ \tilde{\bb}^T]^T$
is the information vector sent by all the $K$ users at time $t=n$
and $b_1$ is the bit sent by the first user at time $t=n$, and
$\bn$ is an $L$-dimensional Gaussian vector with independent
$\sigma^2$-variance components. We maintain the restriction that
$L \geq K$.

Most of the linear multiuser detectors for demodulating the $k$th
user's data bit in (\ref{r}) is in the form of a correlator
followed by a hard limiter, which could be expressed as

\begin{equation}
\begin{array}{rcl}
\hat{b}_k &=& \mbox{sign}\{\bw_k^T\br\}
\end{array} \label{linear}
\end{equation}

\noindent where $\bw_k \in \mathbb{R}^{L\times 1}$ is the linear
representation of multiuser detector. Linear multiuser detectors
can be implemented in a decentralized fashion where only the user
or users of interest need be demodulated.



\section{New Blind Channel Model Using Blind Signature Matrix}

Most classic multiuser detection schemes assume knowledge of the
spreading codes and/or signal-to-noise ratios (SNR) of all active
users that contribute to the received signal. These receivers then
exploit this knowledge to easily achieve optimal or sub-optimal
performance. In practical situations, e.g. at mobile station of
CDMA systems, it is difficult for multiuser receiver to known
other existing users' information. So many blind multiuser
detectors are developed to operate without prior knowledge
regarding other users but use statistic signal processing
techniques to estimate other's information. In this paper, instead
of estimating statistic channel information, we construct a
"faked" blind spreading matrix for the desired user. We then show
that it is possible to detect the desired user's information bits
using this blind spreading matrix. The details are revealed in the
following. Without loss of the generality, assume only the bits
$b_1$ sent by the $1$st user is considered here.

\subsection{Blind Signature Matrix $\bcS$}
At first we construct a new $L\times M$ blind signature matrix
$\bcS$. It is defined by

\begin{equation}
\begin{array}{rcl}
\bcS&=&[\matrix{\bar{\bs}_1&\bar{\bs}_2&\bar{\bs}_3&\ldots&\bar{\bs}_M}]\\
 &=&[\matrix{\bs_1&{\br}_{1}&{\br}_{2}&\ldots&{\br}_{M-1}}]\\
 &=&\left[\matrix{A_1^{-1}\bS\bA\be_1&\bS\bA{\bb}_1&\bS\bA{\bb}_2&\ldots&\bS\bA{\bb}_{M-1}}\right] + {\bN}\\
 &=&\bS\bA\left[\matrix{A_1^{-1}\be_1&{\bb}_1&{\bb}_2&\ldots&{\bb}_{M-1}}\right] + {\bN}\\
 &=&\bS\bA\left[\matrix{A_1\be_1 & \bD }\right] + {\bN}\\
 &=&\bS\bA\bB + {\bN}
\end{array} \label{S}
\end{equation}

\noindent where $L\geq M\geq K$, ${\br}_i$, $i=1,\ 2,\ \ldots,\
M-1$, are the arbitrary previously received independent vectors
and the $K\times 1$ vector ${\bb}_i$ is the corresponding bit
vector for the information sent by all $K$ users. $\bD = [{\bd}\
\tilde{\bD}^{\rm T}]^{\rm T}$, The $(K-1)\times 1 $ vector ${\bd}$
is the bit vector consisting of the known bits previously sent for
the desired user. $\mbox{rank}\{\tilde{\bD}\}=K-1$,
${\bN}=[\mathbf{0}\ \tilde{\bN}]$ and

\begin{equation}
\begin{array}{rcl}
 \bB&=&\left[\matrix{A_1^{-1}\be_1 & \bD }\right]\\
 &=&\left[\matrix{A_1^{-1}\be_1 & \matrix{ {\bd}^{\rm T}\cr\tilde{\bD}} }\right]\\
 &=&\left[\matrix{\bc^{\rm T} \cr \matrix{\mathbf{0}& \tilde{\bD}}
 }\right]\\
 &=&\left[\matrix{A_1^{-1}& \bd^{\rm T} \cr \mathbf{0}& \tilde{\bD} }\right]

\end{array}\label{B}
\end{equation}

\noindent and $\mbox{rank}\{\bB\}=K$.

Using the received signal vector definition (\ref{r}) and the
proposed blind signature matrix $\bcS$ in (\ref{S}), the received
signal vector $\br$ can be expressed as

\begin{equation}
\begin{array}{rcl}
\br&=&\bS\bA\bb + \bn\\
 &=&\bS\bA\bB\bB^{+}\bb + \bn\\
 &=&(\bcS-{\bN})\bB^{+}\bb+ \bn\\
 &=&\bcS\bB^{+}\bb - {\bN}\bB^{+}\bb + \bn\\
 &=&\bcS\bbf + \bar{\bn} \label{rn}
\end{array}
\end{equation}

\noindent where $\bB^{+} = \bB^{\rm T}(\bB\bB^{\rm T})^{-1} $ is
Moor-Penrose general inverse of $\bB$ and $\bbf$ denotes the new
$K \times 1$ detection vector defined by

\begin{equation}
\begin{array}{rcl}
\bbf&=&\left[\matrix{f^1\cr\tilde{\bbf}}\right]\\
 &=&\left[\matrix{A_1^{-1}\be_1&\bD}\right]^{+}\bb\\
 &=&\left[\matrix{A_1^{-1}&{\bd}^T\cr\mathbf{0}&\tilde{\bD}}\right]^{+}\left[\matrix{b_1\cr\tilde{\bb}}\right]
\end{array} \label{DetectorVector}
\end{equation}

\noindent with mean vector $\bm_{\bbf}=\bzero$ and covariance
matrix

\begin{equation}
\begin{array}{rcl}
\bC_{\bbf}&=&E\left\{\bbf \bbf^{\rm T}\right\}.
\end{array} \label{C_f}
\end{equation}

\noindent And $\bar{\bn}$ is the new noise vector defined by

\begin{equation}
\begin{array}{rcl}
\bar{\bn}&=&\bn-{\bN}\bB^{+}\bb
\end{array} \label{new_noise}
\end{equation}

\noindent with mean vector $\bm_{\bar{\bn}}=\bzero$ and covariance
matrix

\begin{equation}
\begin{array}{rcl}
\bC_{\bar{\bn}}&=&E\left\{(\bn-{\bN}\bB^{+}\bb)(\bn-{\bN}\bB^{+}\bb)^{\rm T}\right\}\\
&=& \sigma^{2}\bI+E\left\{{\bN}\bbf\bbf^{\rm T}{\bN}^{\rm T}\right\}\\
&=&\sigma^2\left(1-E\|\tilde{\bbf}\|_2^2\right)\bI\ .
\end{array} \label{var_n}
\end{equation}

\subsection{Information Bit $b_1$ Detection Frame Work}

Before giving blind multiuser detection solutions, we firstly show
a semiblind multiuser detection scheme providing the amplitude,
$A_1$, of the first user is already known. At this time, $b_1$ can
be estimated using following theorem.


\begin{lemma}
The bit $b_1$ sent for the first user at time $t=n$ can be
detected using the following equation.
\begin{equation}
\begin{array}{rcl}
b_1 &=& \mbox{sign}\left\{\bc^{\rm T}\bbf\right\}
\end{array}.
\end{equation} \label{bn_estimation}
\end{lemma}


\begin{proof}

With (\ref{B}), $b_1$ can be estimated by
\begin{equation}
\begin{array}{rcl}
b_1&=&\mbox{sign}\left\{b_1-{\bd}^{\rm T}\tilde{\bD}^{+}\tilde{\bb}+{\bd}^{\rm T}\tilde{\bD}^{+}\tilde{\bb}\right\}\\
&=& \mbox{sign}\left\{\left[\matrix{A_1^{-1}&{\bd}^{\rm
T}}\right]\left[\matrix{A_1b_1-A_1{\bd}^{\rm
T}\tilde{\bD}^{+}\tilde{\bb}\cr\tilde{\bD}^{+}\tilde{\bb}}\right]\right\}\\
&=& \mbox{sign}\left\{\bc^{\rm T}\left[\matrix{A_1&-A_1{\bd}^{\rm
T}\tilde{\bD}^{+}\cr\bzero&\tilde{\bD}^{+}}\right]\left[\matrix{b_1\cr\tilde{\bb}}\right]\right\}\\
&=& \mbox{sign}\left\{\bc^{\rm T}\left[\matrix{A_1&-A_1{\bd}^{\rm
T}\tilde{\bD}^{+}\cr\bzero&\tilde{\bD}^{+}}\right]\bb\right\}
\end{array}\label{b1}
\end{equation}


\noindent On the other hand, we know that

\begin{equation}
\left[A_1^{-1}\be_1\ \matrix{{\bd}^{\rm
T}\cr\tilde{\bD}}\right]\left[A_1\be_1\ \matrix{-A_1{\bd}^{\rm
T}\tilde{\bD}^+\cr\tilde{\bD}^+}\right]
=\left[\matrix{A_1^{-1}&{\bd}^{\rm
T}\cr\bzero&\tilde{\bD}}\right]\left[\matrix{A_1&-A_1{\bd}^{\rm
T}\tilde{\bD}^+\cr\bzero&\tilde{\bD}^+}\right] =\bI \label{psedoB}
\end{equation}

\noindent Using (\ref{psedoB}) and (\ref{B}), (\ref{b1}) can then
be re-written by

\begin{equation}
\begin{array}{rcl}
b_1&=& \mbox{sign}\left\{\bc^{\rm T}\bB^{+}\bb\right\}\\
&=& \mbox{sign}\left\{\bc^{\rm T}\bbf\right\}\\
\end{array}
\end{equation}

\noindent and the detection vector $\bf$ can be re-written by

\begin{equation}
\begin{array}{rcl}
\bbf&=&\left[\matrix{A_1b_1-A_1{\bd}^{\rm
T}\tilde{\bD}^{+}\tilde{\bb}\cr\tilde{\bD}^{+}\tilde{\bb}}\right]
\end{array} \label{DetectorVector2}
\end{equation}


Hence, this lemma is proven.
\end{proof}


We see that the classic multiuser detection model is transferred
into (\ref{rn}) with the information bit vector $\bb$ being
replaced by the detection vector $\bbf$ and the original AWGN
vector $\bn$ being replaced by $\bar{\bn}$ in (\ref{rn}). With
Lemma \ref{bn_estimation}, the bit $b_1$ sent for user 1 can still
be detected with the detection vector $\bbf$ and the previously
detected bits vector $\bc$. In the following, we show that how to
estimate $A_1$ proving we know how to estimate $\bbf$.

Before estimating $A_1$, we construct another blind spreading
matrix $\bcS'$ using another set of received vectors, $\br_m'$. If
we have already known how to estimate the detection vector, we can
get two detection vectors, $\bbf$ and $\bbf'$, corresponding to
these two blind spreading matrices $\bcS$ and $\bcS'$. With $\bbf$
and $\bbf'$, $A_1$ can be estimated using the following theorem.

\begin{lemma}
$A_1$ is the solution to the following equation
\begin{equation}
\begin{array}{rcl}
\bc^{\rm T}\bbf - \bc'^{\rm T}\bbf'&=&0
\end{array}.
\end{equation}


\noindent and it can be estimated by

\begin{equation}
\begin{array}{rcl}
A_1&=&{ f_{1} - f'_{1} \over {\bd'}^{\rm T}\tilde{\bbf}' -
{\bd}^{\rm T}\tilde{\bbf}}
\end{array}
\end{equation}
\end{lemma}


We can see that the performance of this blind multiuser detection
frame work highly depends on the estimation of $\bbf$ and $A_1$
can actually be estimated from $\bbf$ too. In the following,
various algorithms are proposed for estimating $\bbf$ based
different criteria.




Though the PDF of $\bB$ can be determined, the PDF of $\bB^{+}$ is
largely known. This makes it is hard to calculate the closed-form
solution of $\bC_{\bd}$ and $\bC_{\tilde{\bn}}$. However, with
Girko's Law~\cite{Muller,Hanly90}, when $\alpha=(K-1)/(M-1)$ is
fixed, $K$, $M$ $\rightarrow\infty$, the diagonal element of
$\frac{1}{M-1}(\tilde{\bD}\tilde{\bD}^{\rm T})^{+}$ may be
approximated by

\begin{equation}
\begin{array}{rcl}
\frac{1}{M-1}\left[(\tilde{\bD}\tilde{\bD}^{\rm
T})^{+}\right]_{ii}^{-1}&\rightarrow&1-\alpha
\end{array}.
\end{equation}

\noindent So, $\bC_{\bd}$ can be approximated by

\begin{equation}
\begin{array}{rcl}
\bC_{\bd}&=&E\left\{\bd\bd^{\rm T}\right\}\\
&=&E\left\{\left[\matrix{1+\bar{\bd}^T\tilde{\bD}^+\tilde{\bb}\tilde{\bb}^{\rm
T}\tilde{\bD}^{\rm +T}\bar{\bd}&\bzero^{\rm T}\cr \bzero&
\tilde{\bD}^+\tilde{\bb}\tilde{\bb}^{\rm T}\tilde{\bD}^{\rm
+T}}\right]\right\}\\
&\approx&\left[\matrix{M}\right]
\end{array}
\end{equation}


Now, the following result could be very easily researched.



\section{ Least-Squares Detection }
At first, we assume the measurements of $\bcS$ is assumed to be
free of error. Hence, all errors are confined to the received
vector $\br$. Then the following classic LS estimation is
proposed.

 Suppose $\bU^T\bcS\bV=\mathbf{\Sigma}$ is the SVD of $\bcS\in\mathbb{R}^{L\times
 K}$ with $r=rank(\bcS)$. And if $\bU=[\matrix{\bu_1&\bu_2&\ldots&\bu_L}]$,
 $\bV=[\matrix{\bv_1&\bv_2&\ldots&\bv_K}]$, $\mathbf{\Sigma}=diag\{[\matrix{\sigma_1&\ldots\sigma_r&0&\ldots&0}]\}$ and $\br\in \mathbb{R}^{L\times 1}$, then

 \begin{equation}
 \matrix{\bd_{\rm LS}&=&\sum\limits_{i=1}^{r}\frac{\bu_i^T\br}{\sigma_i}\bv_i&=&\bcS^+\br}
 \end{equation}

\noindent minimizes $\|\bcS\bd-\br\|_2$ and has the smallest
2-norm of all minimizers. Moreover
 \begin{equation}
 \matrix{\varepsilon_{\rm LS}^2 &=& \min\limits_{\bx\in\mathbb{R}}\|\bcS\bx-\br\|_2^2 &=& \sum\limits_{i=r+1}^{L}(\bu_i^T\br)^2}
 \end{equation}

So, the least squares estimation of $\bd$ is

\begin{equation}
\begin{array}{rcl}
\bd_{\rm LS} &=& \bcS^+\br\\
 &=&\bd + \bcS^+\tilde{\bn}
\end{array} \label{LS}
\end{equation}

\noindent where the $K\times L$ matrix $\bcS^+$ is the
Moore-Penrose generalized inverse of $\bcS$.

\section{Total Least-Squares Estimation }

The previous LS estimate of the detection vector $\bd$ from
(\ref{rn}), (\ref{DetectorVector}) and (\ref{LS}) is the solution
to
\begin{equation}
\begin{array}{rcl}
\hat{\bd}=\matrix{\min\limits_{\bx}\left\|\br-\bcS\bx\right\|_2}&\mbox{subject
to}&\br\subseteq \mathbb{R}(\bcS)
\end{array}
\label{LSProb}
\end{equation}
where the semi-blind signature matrix $\bcS$ is assumed to be
error-free. However, this assumption is not entirely accurate
according to the definition of $\bcS$ in (\ref{S}) since there is
a noise term, $\bN$.

Furthermore, $\br$ can also be expressed as
\begin{equation}
\begin{array}{rcl}
\br&=&(\bcS-\bN)\bB^{+}\bb + \bn\\
 &=&\hat{\bcS}\bd + \bn
\end{array}
\end{equation}
where  $\hat{\bcS}=\bcS-\bN=\bS\bA\bB$.  The minimization problem
of (\ref{LSProb}) can then be transformed into the following TLS
problem:
\begin{equation}
\begin{array}{rcl}
\left[\hat{\bcS},\ \bx\right]&=&\matrix{\min\limits_{\bar{\bcS},\
\bx}\left\|\left[ \matrix{\bcS&\br} \right] - \left[
\matrix{\bar{\bcS}& \bar{\bcS}\bx}\right]\right\|_2}
\end{array},
\label{TLSProb}
\end{equation}
subject to $\br\subseteq\mathbb{R}(\bar{\bcS})$.

 Let $\bcS=\bU^{'}\mathbf{\Sigma}^{'}\bV^{'T}$ and
$[\bcS\ \br]=\bU\mathbf{\Sigma}\bV^T$ be the SVD of $\bcS$ and
$[\bcS\ \br]$, respectively. If $\sigma_K^{'}
> \sigma_{K+1}$, then
\begin{equation}
\bd_{\rm TLS} =
\left(\bcS^T\bcS-\sigma_{K+1}^2\bI\right)^{-1}\bcS^T\br
\end{equation}
and
\begin{equation}
\begin{array}{rcl}
\varepsilon_{\rm TLS}^{2}&=&\min\limits_{\bx\in
\mathbb{R}^{K\times
1}}\|\bcS\bx-\br\|_2^2 \\
 &=& \sigma_{K+1}^2\left[1+\sum_{i+1}^{K}\frac{\left(\bu_i^{'T}\br\right)^2}
{\sigma_i^{'2}-\sigma_{K+1}^2}\right]
\end{array}
\end{equation}
where $\bU=[\bu_1\ \bu_2\ \ldots\ \bu_L]$, $\bV=[\bv_1\ \bv_2\
\ldots\ \bv_{K+1}]$, $\mathbf{\Sigma}={\rm diag}\{[\sigma_1\
\sigma_2 \ldots\ \sigma_{K+\min\limits\{L-K,\ 1\}}]\}$ and
$\bU^{'}=[\bu_1^{'}\ \bu_2^{'}\ \ldots\ \bu_L^{'}]$,
 $\bV^{'}=[\bv_1^{'}\ \bv_2^{'}\ \ldots\ \bv_{K}^{'}]$,
 $\mathbf{\Sigma}^{'}={\rm diag}\{[\sigma_1^{'}\ \sigma_2^{'}\ \ldots\ \sigma_K^{'}]\}$.

\section{Mixed LS/TLS Estimation}

In the LS problem of (\ref{LSProb}), it assumed the semi-blind
signature matrix $\bcS$ is error-free. Again, this assumption is
not completely accurate. In the TLS problem of (\ref{TLSProb}), it
assumed that in each column of the semi-blind signature matrix,
$\bcS$, some noise or error exists.  This assumption also is not
complete. Though there exists a noise or error matrix $\bN$ in
$\bcS$ from (\ref{S}), its first column is exactly known to be
noise-free or error-free.  Hence, to maximize the estimation
accuracy of the detection vector $\bd$, it is natural to require
that the corresponding columns of $\bcS$ be unperturbed since they
are known exactly. The problem  of estimating the detection vector
$\bd$ can then be transformed into the following MLS problem by
considering (\ref{LSProb}) and (\ref{TLSProb}):
\begin{equation}
\begin{array}{rcl}
\matrix{\min\limits_{\bar{\bcS},\
\bx}\left\|\left[\matrix{\tilde{\bcS}&\br}\right]-\left[\matrix{\bar{\bcS}&[A_1\bs_1\
 \bar{\bcS}]\bx}\right]\right\|_{2} }
\end{array}\label{MLSProb}
\end{equation}
subject to $\br\subseteq\mathbb{R}([A_1\bs_1\ \bar{\bcS}])$.  The
following lemma outlines the MLS solution.

Consider the MLS problem in (\ref{MLSProb}) and perform the
Householder transformation $Q$ on the matrix $[\matrix{\bcS&\br}]$
so that
\begin{equation}
\begin{array}{rcl}
Q^{\rm
T}[\matrix{A_1\bs_1&\bar{\bcS}&\br}]&=&\left[\matrix{R_{11}&\bR_{12}&R_{1r}\cr
\mathbf{0}&\bR_{22}&\bR_{2r}}\right]
\end{array}
\end{equation}
where $R_{11}\neq 0$, $\bR_{12}$ is a $1\times (M-1)$ vector,
$\bR_{22}$ is a $(L-1)\times (M-1)$ matrix and $\bR_{2r}$ is a
$(L-1)\times 1$ vector.

Denote $\sigma'$ as the smallest singular value of $\bR_{22}$ and
$\sigma$ as the smallest singular value of
$[\matrix{\bR_{22}&\bR_{2r}}]$. If $\sigma'>\sigma$, then the MLS
solution uniquely exists and is given by
\begin{equation}
\begin{array}{rcl}
\bd_{\rm
MLS}&=&\left(\bcS^T\bcS-\sigma^2\left[\matrix{0&\mathbf{0}\cr\mathbf{0}&\mathbf{I}_{M-1}}\right]\right)^{-1}\bcS^T\br
\end{array}.
\end{equation}


\section{ Best Linear Unbiased Estimation}
We begin by assuming the following linear structure
${\bd}_{BLU}=\bG\br$ for this so-called best linear unbiased
estimator (BLUE). Matrix $\bG$ is designed such that: (1) $\bcS$
must be deterministic, and (2) $\tilde{\bn}$ must be zero mean
with positive definite known covariance matrix
$\bC_{\tilde{\bn}}$. (3) ${\bd}_{BLU}$ is an unbiased estimator of
$\bd$, and (4) the error variance for each of the $M$ parameters
is minimized as

\begin{equation}
\begin{array}{rcl}
{\bd}_{\rm BLU}&=&\min\limits_{\bG} var(\bG\br)
\end{array}
\end{equation}

In this way, ${\bd}_{BLU}$ will be unbiased and efficient (within
the class of linear estimators) by design. The resulting best
linear unbiased estimator is (Gauss-Markov Theorem):

\begin{equation}
\begin{array}{rcl}
{\bd}_{\rm BLU}$=$(\bcS^{\rm
T}\bC_{\tilde{\bn}}^{-1}\bcS)^{-1}\bcS^{\rm
T}\bC_{\tilde{\bn}}^{-1}\br
\end{array} \label{BLUE}
\end{equation}

\noindent with the covariance matrix of ${\bd}_{BLU}$ is

\begin{equation}
\begin{array}{rcl}
{\bC}_{\bd_{\rm BLU}}$=$(\bcS^{\rm
T}\bC_{\tilde{\bn}}^{-1}\bcS)^{-1}
\end{array}.
\end{equation}

\noindent Since the above data are of Gaussian distributions, the
BLUE (\ref{BLUE}) is also the minimum variance unbiased
estimation.

With (\ref{var_n}), (\ref{BLUE}) can be simplified into
\begin{equation}
\begin{array}{rcl}
{\bd}_{\rm BLU}$=$(\bcS^{\rm T}\bcS)^{-1}\bcS^{\rm T}\br
\end{array}.
\end{equation}



\section{Minimum Mean-Squared Estimation}
Given measurements $\br$, the mean-sqaured estimator (MSE) of
$\bd$, ${\bd}_{MS} = f( \br )$, minimizes the mean-squared error
$J_{MSE}=E\{||\bd-{\bd}||_2^2\}$. The function $f(\br)$ may be
nonlinear or linear and its exact structure is determined by
minimizing $J_{\bd}$. When $\bd$ and $\br$ are jointly Gaussian,
the linear estimator that minimizes the mean-sqared error is
(Bayesian Gauss-Markov Theorem)

\begin{equation}
\begin{array}{rcl}
{\bd}_{\rm MS} &=& (\bC_{\bd}^{-1}+\bcS^{\rm
T}\bC_{\tilde{\bn}}^{-1}\bcS)^{-1}\bcS^{\rm
T}\bC_{\tilde{\bn}}^{-1}\br
\end{array} \label{MSE}
\end{equation}

\noindent with the variance matrix of ${\bd}_{MS}$ is

\begin{equation}
\begin{array}{rcl}
\bC_{{\bd}_{\rm MS}} &=& (\bC_{\bd}^{-1}+\bcS^{\rm
T}\bC_{\tilde{\bn}}^{-1}\bcS)^{-1}
\end{array}.
\end{equation}

\section{Kalman Filter Estimation}
\begin{equation}
\begin{array}{rcl}
\mathbf{x}[n]&=&\mathbf{P}\mathbf{x}[n-1] +
\mathbf{Q}\mathbf{u}[n]
\end{array} \label{KFE_SF}
\end{equation}

\begin{equation}
\begin{array}{rcl}
\mathbf{y}[n]&=&\mathbf{H}\mathbf{x}[n] + \mathbf{w}[n]
\end{array} \label{KFE_SF}
\end{equation}




\bibliographystyle{unsrt}
\bibliography{FastBDD}

\end{document}
